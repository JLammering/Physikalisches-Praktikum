\section{Diskussion}
\label{sec:Diskussion}

Das Ablesen der Sättigungsströme kann bei geringerem Heizstrom
in diesem Messbereich besser erfolgen. Bei $I_2$
muss schon mit einer hohen Unsicherheit abgelesen werden.

Der Exponent $b$ des Raumladungsgesetzes kann gut bestimmt werden.
Auch die Fitkurve nähert den Verlauf der Messwerte gut an.

Der Fit der Messwerte im Anlaufstromgebiet funktioniert ebenfalls
sehr gut. Die Kathodentemperatur hat einen vernünftigen Wert.

Bei dieser Messung muss beachtet werden, dass die $N_\g{zu}$ zugehörige
Spannung nur sehr ungenau abgelesen werden kann. Die bestimmten
Kathodentemperaturen liegen im gleichen Bereich wie die vorher aus den
Messwerten im Anlaufstromgebiet bestimmte Kathodentemperatur.

Die Austrittsarbeit konnte annehmbar gut bestimt werden. 
