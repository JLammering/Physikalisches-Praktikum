\section{Auswertung}
\label{sec:Auswertung}

Die genutzten Naturkonstanten \cite{Codata} sind:

\begin{align*}
  e &= \SI{1.6021766208(98)e-19}{\coulomb}\\
  k &= \SI{1.38064852(79)e-23}{\joule\per\kelvin}\\
  h &= \SI{6.626070040(81)e-34}{\joule\second}\\
  m_\g{e} &= \SI{9.10938356(11)e-31}{\kilo\gram}.
\end{align*}

\subsection{Kennlinienschar der Hochvakuumdiode}
\label{sec:kennlinien}

\begin{table}[h]
  \centering
  \begin{tabular}{S S S S S S}
    \toprule
    {$U\:/\:\si{\milli\volt}$} & {$I_1 \:/\: \si{\milli\ampere}$} & {$I_2 \:/\: \si{\milli\ampere}$} & {$I_3 \:/\: \si{\milli\ampere}$}
     & {$I_4 \:/\: \si{\milli\ampere}$} & {$I_5 \:/\: \si{\milli\ampere}$}\\
    \midrule
    5 & 0.014 & 0.014 & 0.014 & 0.013 & 0.009\\
    10 & 0.031 & 0.030 & 0.036 & 0.031 & 0.027\\
    15 & 0.050 & 0.050 & 0.060 & 0.055 & 0.045\\
    20 & 0.076 & 0.075 & 0.087 & 0.080 & 0.066\\
    25 & 0.105 & 0.107 & 0.121 & 0.105 & 0.084\\
    30 & 0.138 & 0.136 & 0.148 & 0.136 & 0.106\\
    35 & 0.173 & 0.170 & 0.180 & 0.165 & 0.126\\
    40 & 0.231 & 0.208 & 0.216 & 0.195 & 0.153\\
    45 & 0.255 & 0.248 & 0.247 & 0.226 & 0.175\\
    50 & 0.297 & 0.294 & 0.284 & 0.254 & 0.197\\
    55 & 0.349 & 0.336 & 0.326 & 0.285 & 0.216\\
    60 & 0.402 & 0.382 & 0.363 & 0.319 & 0.234\\
    70 & 0.509 & 0.488 & 0.445 & 0.379 & 0.262\\
    80 & 0.629 & 0.595 & 0.537 & 0.442 & 0.284\\
    90 & 0.743 & 0.698 & 0.631 & 0.497 & 0.298\\
    100 & 0.842 & 0.794 & 0.709 & 0.540 & 0.309\\
    110 & 0.970 & 0.910 & 0.791 & 0.582 & 0.316\\
    120 & 1.10 & 1.02 & 0.867 & 0.619 & 0.322\\
    130 & 1.23 & 1.14 & 0.940 & 0.648 & 0.326\\
    140 & 1.38 & 1.26 & 1.01 & 0.666 & 0.329\\
    150 & 1.51 & 1.38 & 1.07 & 0.678 & 0.331\\
    160 & 1.66 & 1.51 & 1.12 & 0.688 & 0.333\\
    170 & 1.79 & 1.61 & 1.19 & 0.701 & 0.333\\
    180 & 1.93 & 1.71 & 1.24 & 0.715 & 0.329\\
    190 & 2.07 & 1.81 & 1.28 & 0.730 & 0.330\\
    200 & 2.20 & 1.90 & 1.31 & 0.732 & 0.334\\
    210 & 2.34 & 2.00 & 1.34 & 0.737 & 0.338\\
    220 & 2.47 & 2.09 & 1.37 & 0.741 & 0.339\\
    230 & 2.60 & 2.18 & 1.38 & 0.744 & 0.340\\
    240 & 2.73 & 2.26 & 1.40 & 0.748 & 0.341\\
    250 & 2.85 & 2.34 & 1.42 & 0.751 & 0.343\\
    \bottomrule
  \end{tabular}
  \caption{Der gemessene Anodenstrom $I$ in Abhängigkeit von der Spannung $U$.}
  \label{tab:kennlinien}
\end{table}

In Tabelle \ref{tab:kennlinien} sind die Messwerte zur Bestimmung der Kennlinien
bei fünf verschiedenen Heizströmen:
\begin{align*}
  I_{\text{H}_1} &= \SI{2.5}{\ampere} & I_{\text{H}_2} &= \SI{2.4}{\ampere} & I_{\text{H}_3} &= \SI{2.3}{\ampere} &
  I_{\text{H}_4} &= \SI{2.2}{\ampere} & I_{\text{H}_5} &= \SI{2.1}{\ampere}\\
  \intertext{und den zugehörigen Spannungen}
  U_{\text{H}_1} &= \SI{6}{\volt} & U_{\text{H}_2} &= \SI{6}{\volt} & U_{\text{H}_3} &= \SI{5.5}{\volt} &
  U_{\text{H}_4} &= \SI{5}{\volt} & U_{\text{H}_5} &= \SI{5}{\volt}
\end{align*}
eingetragen.

Grafisch dargestellt sind sie in Abbildung \ref{fig:kennlinien}.

\begin{figure}
  \centering
  \includegraphics[width = \textwidth]{build/kennlinien.pdf}
  \caption{Die Kennlinien der Diode bei fünf unterschiedlichen Heizströmen.}
  \label{fig:kennlinien}
\end{figure}

Hier wurde der Sättigungsstrom $I_\text{S}$ abgelesen:
\begin{align*}
  I_{\text{S}_2} &= \SI{2.6}{\milli\ampere} & I_{\text{S}_3} &= \SI{1.5}{\milli\ampere} & I_{\text{S}_4} &= \SI{0.76}{\milli\ampere} & I_{\text{S}_5} &= \SI{0.35}{\milli\ampere}.
\end{align*}
Der Sättigungsstrom für $I_1$ kann nicht bestimmt werden, da die Messwerte in diesem
Messbereich noch kein ausreichendes asymptotisches Verhalten zeigen. Für $I_2$ kann er nur
ungenau bestimmt werden.
\FloatBarrier

\subsection{Gültigkeitsbereich des Langmuir-Schottkyschen Raumladungsgesetzes}

Für den Gültigkeitsbereich des Raumladungsgesetzes wurden alle Messwerte von $I_1$
genutzt bis einschließlich Spannung $U = \SI{210}{\volt}$.
Nun werden für die Ausgleichsfunktion:
\begin{equation}
  I = a \cdot U^b
\end{equation}
mittels Ausgleichsrechnung die beiden Parameter $a$ und $b$ bestimmt.

Die ausgewählten Messwerte und die bestimmte Ausgleichskurve sind in Abbildung
\ref{fig:raumladung} zu sehen.
\begin{figure}
  \centering
  \includegraphics[width = \textwidth]{build/raumladung.pdf}
  \caption{Das Raumladungsgebiet der Anode bei höchstem Heizstrom.}
  \label{fig:raumladung}
\end{figure}

Bestimmt wurden die Fitparameter:
\begin{align*}
  a &= \SI{1.38(8)e-6}{\ampere\per\volt}\\
  b &= \num{1.39(1)}.
\end{align*}

Die Abweichung des Exponenten $b$ vom theoretischen Wert $\num{1.5}$
beträgt:
\begin{equation*}
  \Delta b = \SI{7.1}{\percent}.
\end{equation*}
\FloatBarrier

\subsection{Bestimmung der Kathodentemperatur aus dem Anlaufstromgebiet}

Zunächst wird die angezeigte Spannung korrigiert, da an dem in Reihe geschalteten
Amperemeter eine Spannung abfällt. Der Innenwiderstand dort beträgt $R_\g{i} = \SI{1}{\mega\ohm}$.
Nach dem Ohmschen Gesetz folgt diese Formel:
\begin{equation}
  U_\text{korrigiert} = U_\text{angezeigt} - R_\g{i} \cdot I.
\end{equation}

Die gemessenen und korrigierten Werte für $U$ sowie die Stromstärken sind in Tabelle \ref{tab:anlauf}
eingetragen.

\begin{table}[h]
  \centering
  \begin{tabular}{S S S}
    \toprule
    {$U\:/\:\si{\volt}$} & {$U_\text{korr.}\:/\:\si{\volt}$} & {$I\:/\:\si{\nano\ampere}$}\\
    \midrule
    -0 & 0.013 & 13\\
    -0.04 & -0.029 & 11\\
    -0.08 & -0.0709 & 9.1\\
    -0.12 & -0.1125 & 7.5\\
    -0.16 & -0.1539 & 6.1\\
    -0.2 & -0.195 & 5.0\\
    -0.24 & -0.236 & 4.0\\
    -0.28 & -0.2768 & 3.2\\
    -0.32 & -0.3174 & 2.6\\
    -0.36 & -0.3579 & 2.1\\
    -0.40 & -0.3983 & 1.7\\
    -0.44 & -0.4386 & 1.4\\
    -0.48 & -0.4789 & 1.1\\
    -0.52 & -0.5191 & 0.95\\
    -0.56 & -0.5593 & 0.75\\
    -0.6 & -0.5994 & 0.60\\
    -0.64 & -0.6395 & 0.50\\
    -0.68 & 0.6796 & 0.40\\
    -0.72 & -0.7196 & 0.39\\
    -0.76 & -0.7597 & 0.33\\
    -0.8 & -0.7997 & 0.28\\
    -0.84 & -0.8398 & 0.23\\
    -0.88 & -0.8798 & 0.20\\
    -0.92 & -0.9198 & 0.17\\
    -0.96 & -0.9599 & 0.15\\
    \bottomrule
  \end{tabular}
  \caption{Werte des Anlaufstromgebiets.}
  \label{tab:anlauf}
\end{table}

Die zum Ausgleich genutzte Kurvengleichung lautet:
\begin{equation}
  I = c e^{d\cdot U}.
  \label{eqn:ausgleichanlauf}
\end{equation}

\begin{figure}
  \centering
  \includegraphics[width = \textwidth]{build/anlauf.pdf}
  \caption{Das Anlaufstromgebiet der Diode.}
  \label{fig:anlauf}
\end{figure}

In Abbildung \ref{fig:anlauf} ist die gemessene Stromstärke gegen die korrigierte Spannung
aufgetragen. Außerdem ist die Ausgleichskurve eingezeichnet.

Die sich aus der Ausgleichsrechnung ergebenden Fitparameter betragen:
\begin{align*}
  c &= \SI{12.53(8)}{\nano\ampere}\\
  d &= \SI{4.83(5)}{\per\volt}.
\end{align*}

Aus Gleichung \eqref{eqn:Anlaufstromst} folgt:
\begin{align*}
  d = \frac{e}{k T}\\
  \intertext{und damit: }
  T = \frac{e}{k d}.\\
  \intertext{Eingesetzt ergibt sich:}
  T = \SI{2402(26)}{\kelvin}.
\end{align*}
\FloatBarrier

\subsection{Bestimmung der Kathodentemperatur aus der Leistungsbilanz}

Mit den Messwerten aus Kapitel \ref{sec:kennlinien} wird zunächst die zugeführte
Leistung $N_\text{zu}$ nach Formel \eqref{eqn:Nzu} bestimmt.
\begin{equation}
  N_\text{zu} = U_\g{f} I_\g{f}
  \label{eqn:Nzu}
\end{equation}
Dann kann die Kathodentemperatur $T$ nach Gleichung \eqref{eqn:kathodentemperatur}
bestimmt werden.
\begin{equation}
  T = \sqrt[4]{\frac{N_\g{zu} - N_\g{WL}}{f \eta \sigma}}
  \label{eqn:kathodentemperatur}
\end{equation}
Angegeben sind \cite{anleitung}:
\begin{align*}
  f &= \SI{0.35}{\centi\meter\squared} & \sigma &= \SI{5.7e-12}{\watt\per\centi\meter\squared\kelvin\tothe{4}} &
  \eta &= 0.28 & N_\g{WL} &= \SI{1}{\watt}.
\end{align*}
Die bestimmten Leistungen $N_\g{zu}$ und Kathodentemperaturen sind in Tabelle \ref{tab:kathodentemperatur}
zu finden.

\begin{table}[h]
  \centering
  \begin{tabular}{S S}
    \toprule
    {$N_\g{zu}\:/\:\si{\watt}$} & {$T\:/\:\si{\kelvin}$}\\
    \midrule
    15 & 2237.5\\
    14.4 & 2213.1\\
    12.65 & 2137.0\\
    11 & 2057.0\\
    10.5 & 2030.7\\
    \bottomrule
  \end{tabular}
  \caption{Die zugeführte Leistung des Heizstroms sowie die Ergebnisse der Kathodentemperatur.}
  \label{tab:kathodentemperatur}
\end{table}

\FloatBarrier

\subsection{Bestimmung der Austrittsarbeit}

Gleichung \eqref{eqn:richardson} lässt sich umstellen zu:
\begin{equation}
  e \phi = - k T \ln \left(\frac{I_\g{S} h^3}{4 \pi f e m_\g{e} k^2 T^2}\right).
\end{equation}

Mit den abgelesen Sättigungsströmen und den bestimmten Kathodentemperaturen
lässt sich nun die Austrittsarbeit $e \phi$ bestimmen. Die Ergebnisse sind
 in Tabelle \ref{tab:Austrittsarbeit} zu
sehen.
Von der Angabe der Fehler wurde abgesehen, da sie sehr klein, da
sie nur aus der sehr genauen Angabe der physikalischen Konstanten
resultieren.
Der Literaturwert zur Austrittsarbeit von Wolfram beträgt \cite{Spektrum}:
\begin{equation*}
  e\phi_\g{lit.} = \SI{4.54}{\electronvolt}.
\end{equation*}

\begin{table}[h]
  \centering
  \begin{tabular}{S S S S}
    \toprule
    {Nr. der Kennlinie} & {$I_\g{S}\:/\:\si{\milli\ampere}$} & {$T\:/\:\si{\kelvin}$}
    & {$e \phi\:/\:\si{\electronvolt}$}\\
    \midrule
    2 & 2.6 & 2213.1 & 4.79\\
    3 & 1.5 & 2137.0 & 4.71 \\
    4 & 0.76 & 2057 & 4.64\\
    5 & 0.35 & 2030.7 & 4.71\\
    \bottomrule
  \end{tabular}
  \caption{Die aus dem Sättigungsstrom und der Kathodentemperatur bestimmte Austrittsarbeit.}
  \label{tab:Austrittsarbeit}
\end{table}

Daraus wird der Mittelwert bestimmt:
\begin{align*}
  e \phi_\g{mittel.} &= \SI{4.71 +- 0.05}{\electronvolt};\\
  \intertext{die Abweichung vom Literaturwert beträgt:}
  \Delta e\phi_\g{mittel} &= \SI{3.8}{\percent}.
\end{align*}
