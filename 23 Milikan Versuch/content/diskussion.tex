\section{Diskussion}
\label{sec:Diskussion}

Der Literaturwert der Elementarladung \cite{elementary} lautet
\begin{align}
  e_\text{0,lit} \approx \SI{1.602 176 6208(98)e-19}{\coulomb}.
\end{align}
Die berechneten Werte für die Elementarladung ohne und mit Cunninghamkorrektur
sind
\begin{align}
  e_\text{0,unkorr} & = \SI{1.67(6)e-19}{\coulomb} \\
  e_\text{0,korr} & = \SI{1.56(7)e-19}{\coulomb}.
\end{align}
Daraus folgen die Abweichungen zwischen den berechneten Werten und dem Literaturwert
\begin{align}
  \delta e_\text{0,unkorr} & = \SI{-4.2}{\percent} \\
  \delta e_\text{0,korr} & = \SI{2.6}{\percent}.
\end{align}
Beide Werte sind sehr nahe am Literaturwert. Die Öltröpfchen-Methode ist
also eine sehr gute Methode, die Elementarladung zu bestimmen. Werden mehr
Öltropfen betrachtet, so kann die Elementarladung auch ohne Hilfswert sehr
genau angegeben werden.
Der korrigierte Wert liegt sichtbar näher an dem Literaturwert als der
nicht-korrigierte. Die Cunningham-Korrektur ist also sinnvoll, um genauere
Werte zu bestimmen.
Der Literaturwert der Avogadro-Konstante \cite{avogadro} lautet
\begin{align}
  N_\text{A,lit} = \SI{6.022140857(74)e23}{\per\mol}.
\end{align}
Die berechneten Werte für die Avogadro-Konstante mit unkorrigierter und
korrigierter Elementarladung sind
\begin{align}
    N_\text{A,unkorr} & = \SI{5.77(19)e23}{\per\mol}\\
    N_\text{A,korr} & = \SI{6.20(27)e23}{\per\mol}.
\end{align}
Daraus ergeben sich die Abweichungen vom Literaturwert
\begin{align}
  \delta N_\text{A,unkorr} & = \SI{4.2}{\percent} \\
  \delta N_\text{A,korr} & = \SI{-3.0}{\percent}
\end{align}
Auch diese Abweichungen sind gering. Wie schon bei der Elementarladung liegt
der korrigierte Wert näher am Literaturwert, was wiederum für die
Cunningham-Korrektur spricht.
