\section{Auswertung}
\label{sec:Auswertung}

Aus den gemessenen Widerständen
\begin{align}
  R_1 = \SI{1728}{\kilo\ohm}
\end{align}
bei den ersten fünf Öltropfen und
\begin{align}
  R_2 = \SI{1706}{\kilo\ohm}
\end{align}
bei den letzten beiden folgen mit Tabelle \ref{tab:thermistor} die Temperaturen
\begin{align}
  T_1 & = \SI{31}{\celsius} \\
  T_2 & = \SI{32}{\celsius}.
  \label{eqn:gemTemp}
\end{align}

\subsection{Auswahl der Öltropfen}

Die Messwerte der Zeiten, in der die Öltropfen jeweils die Strecke
\begin{align}
  s = \SI{1}{\milli\meter}
\end{align}
zurücklegen, sind in den Tabellen \ref{tab:Zeiten1}, \ref{tab:Zeiten2},
\ref{tab:Zeiten3} und \ref{tab:Zeiten4} aufgelistet.

\begin{table}[h]
  \centering
  \begin{tabular}{S S S}
    \toprule
    {$t_0/\si{\second}$} & {$t_\text{ab}/\si{\second}$} & {$t_\text{auf}/\si{\second}$}\\
    \midrule
    29.64    & 7.92 & 20.92 \\
    \text{ } & 8.49 & 22.53 \\
    \text{ } & 7.81 & 22.09 \\
    \addlinespace[0.3cm]
    \to & 8.1 \pm 0.4 & 21.8 \pm 0.8 \\
    \addlinespace[0.5cm]
    35.35    & 3.78 & 4.60 \\
    \text{ } & 4.13 & 5.43 \\
    \text{ } & 4.24 & 5.66 \\
    \text{ } & 4.09 & 5.90 \\
    \text{ } & 4.38 & 4.92 \\
    \addlinespace[0.3cm]
    \to & 4.12 \pm 0.22 & 5.3 \pm 0.5 \\
    \addlinespace[0.2cm]
    \bottomrule
  \end{tabular}
  \caption{Zeitenmesswerte und daraus resultierende Mittelwerte der Öltröpfchen
  bei einer Spannung $U_1=\SI{313}{\volt}$ und
  einer Temperatur $T_1=\SI{31}{\degree}$.}
  \label{tab:Zeiten1}
\end{table}

\begin{table}[h]
  \centering
  \begin{tabular}{S S S}
    \toprule
    {$t_0/\si{\second}$} & {$t_\text{ab}/\si{\second}$} & {$t_\text{auf}/\si{\second}$}\\
    \midrule
    61.61    & 8.18 & 9.93  \\
    \text{ } & 7.97 & 9.64  \\
    \text{ } & 8.00 & 10.38 \\
    \text{ } & 8.63 & 9.81  \\
    \text{ } & 7.95 & 9.40  \\
    \text{ } & 8.78 & 10.03 \\
    \text{ } & 9.93 & 8.96  \\
    \text{ } & 8.24 & 10.36 \\
    \addlinespace[0.3cm]
    \to & 8.5 \pm 0.7 & 9.8 \pm 0.5  \\
    \addlinespace[0.5cm]
    48.53 & 12.50 & 27.01\\
    \text{ } & 12.76 & 28.76\\
    \text{ } & 9.30  & 26.40\\
    \text{ } & 9.38  & 16.50\\
    \addlinespace[0.3cm]
    \to & 11.0 \pm 1.9 & 24.7 \pm 5.5 \\
    \addlinespace[0.2cm]
    \bottomrule
  \end{tabular}
  \caption{Zeitenmesswerte und daraus resultierende Mittelwerte der Öltröpfchen
  bei einer Spannung $U_2=\SI{249}{\volt}$ und
  einer Temperatur $T_1=\SI{31}{\degree}$.}
  \label{tab:Zeiten2}
\end{table}

\begin{table}[h]
  \centering
  \begin{tabular}{S S S}
    \toprule
    {$t_0/\si{\second}$} & {$t_\text{ab}/\si{\second}$} & {$t_\text{auf}/\si{\second}$}\\
    \midrule
    31.76    & 5.60 & 9.40 \\
    \text{ } & 5.95 & 9.40 \\
    \text{ } & 5.36 & 8.16 \\
    \text{ } & 5.55 & 8.81 \\
    \text{ } & 5.67 & 8.43 \\
    \addlinespace[0.3cm]
    \to & 5.63 \pm 0.21 & 8.8 \pm 0.6 \\
    \addlinespace[0.2cm]
    \bottomrule
  \end{tabular}
  \caption{Zeitenmesswerte und daraus resultierende Mittelwerte der Öltröpfchen
  bei einer Spannung $U_3=\SI{272}{\volt}$ und
  einer Temperatur $T_1=\SI{31}{\degree}$.}
  \label{tab:Zeiten3}
\end{table}

\begin{table}[h]
  \centering
  \begin{tabular}{S S S}
    \toprule
    {$t_0/\si{\second}$} & {$t_\text{ab}/\si{\second}$} & {$t_\text{auf}/\si{\second}$}\\
    \midrule
    56.96    & 5.13 & 6.23 \\
    \text{ } & 4.26 & 5.73 \\
    \text{ } & 4.55 & 5.63 \\
    \text{ } & 4.18 & 5.09 \\
    \text{ } & 3.86 & 5.53 \\
    \text{ } & 4.24 & 5.09 \\
    \text{ } & 4.41 & 5.63 \\
    \text{ } & 3.70 & 4.52 \\
    \text{ } & 3.49 & 4.49 \\
    \text{ } & 3.36 & 3.90 \\
    \addlinespace[0.3cm]
    \to & 4.1 \pm 0.5 & 5.2 \pm 0.7 \\
    \addlinespace[0.5cm]
    40.73    & 2.93 & 3.23 \\
    \text{ } & 3.21 & 3.27 \\
    \text{ } & 2.81 & 3.35 \\
    \text{ } & 2.84 & 3.46 \\
    \text{ } & 2.90 & 3.38 \\
    \text{ } & 2.83 & 3.21 \\
    \text{ } & 2.60 & 3.46 \\
    \addlinespace[0.3cm]
    \to & 2.87 \pm 0.18 & 3.34 \pm 0.10 \\
    \addlinespace[0.2cm]
    \bottomrule
  \end{tabular}
  \caption{Zeitenmesswerte und daraus resultierende Mittelwerte der Öltröpfchen
  bei einer Spannung $U_4=\SI{287}{\volt}$ und
  einer Temperatur $T_2=\SI{32}{\degree}$.}
  \label{tab:Zeiten4}
\end{table}

\FloatBarrier

In Tabelle \ref{tab:Geschw} sind die daraus resultierenden Geschwindigkeiten
und der Quotient
\begin{align}
  \delta v = \frac{2 v_0}{v_\text{ab}-v_\text{auf}} \approx 1
  \label{eqn:QuotientG}
\end{align}
für jedes betrachtete Öltröpfchen dargestellt.
Der Quotient ist bei allen, ausser dem dritten Öltröpfchen im Bereich
\begin{align}
  \delta v_\text{ok} = \num{1.0(3)}.
\end{align}
Beim dritten Öltröpfchen ist die Abweichung deutlich größer, weshalb dieses
in den folgenden Berechnungen vernachlässigt wird.

\begin{table}[h]
  \centering
  \begin{tabular}{S S S[table-format=2.1] @{${}\pm{}$} S[table-format=1.1]
    S[table-format=2.1] @{${}\pm{}$} S[table-format=1.1] S @{${}\pm{}$} S}
    \toprule
    {Nr.} & {$v_0/\SI{e-5}{\meter\per\second}$} & \multicolumn{2}{c}{$v_\text{ab}/\SI{e-5}{\meter\per\second}$} &
    \multicolumn{2}{c}{$v_\text{auf}/\SI{e-5}{\meter\per\second}$} & \multicolumn{2}{c}{$\delta v$} \\
    \midrule
    1 & 3.37 & 12.4 & 0.6 & 4.6  & 0.2 & 0.9 & 0.1 \\
    2 & 2.83 & 24.2 & 1.3 & 18.9 & 1.9 & 1.1  & 0.4 \\
    3 & 1.62 & 11.8 & 0.9 & 10.2 & 0.5 & 2.0  & 1.3 \\
    4 & 2.06 & 9.1  & 1.6 & 4.1  & 0.9 & 0.8 & 0.3 \\
    5 & 3.15 & 17.8 & 0.7 & 11.3 & 0.7 & 1.0 & 0.2 \\
    6 & 1.76 & 24.3 & 3.1 & 19.3 & 2.6 & 0.7  & 0.6 \\
    7 & 2.46 & 34.8 & 2.2 & 30.0 & 0.9 & 1.0  & 0.5 \\
    \bottomrule
  \end{tabular}
  \caption{Geschwindigkeiten der Öltröpfchen und Verhältnis
  \eqref{eqn:QuotientG}. Das dritte Öltröpfchen wird aufgrund der Abweichung
  bei den Ausgleichsrechnungen vernachlässigt.}
  \label{tab:Geschw}
\end{table}

\FloatBarrier

\subsection{Unkorrigierte Elementarladung}
\label{sec:unkoel}

Die unkorrigierten Viskositäten der Luft ergeben sich mit den Temperaturen
\eqref{eqn:gemTemp} und dem Graphen \ref{fig:Visko}
\begin{align}
  \eta_\text{L,1} & = \SI{1.875e-5}{\newton\second\per\square\meter} \\
  \eta_\text{L,2} & = \SI{1.881e-5}{\newton\second\per\square\meter}.
\end{align}
Mit der Formel \eqref{eqn:Radius} folgen die unkorrigierten Radien der Öltropfen
in Tabelle \ref{tab:Radien}.

\begin{table}[h]
  \centering
  \begin{tabular}{S S @{${}\pm{}$} S}
    \toprule
    {Nr.} & \multicolumn{2}{c}{$r_\text{unkorr}/\SI{e-7}{\meter}$}\\
    \midrule
    1 & 6.2 & 0.2 \\
    2 & 5.1 & 1.1 \\
    3 & 2.8 & 0.9 \\
    4 & 5.0 & 0.9 \\
    5 & 5.6 & 0.4 \\
    6 & 4.9 & 2.0 \\
    7 & 4.9 & 1.2 \\
    \bottomrule
  \end{tabular}
  \caption{Berechnete Radien der Öltröpfchen.}
  \label{tab:Radien}
\end{table}

\FloatBarrier

Außerdem ergeben sich mit \eqref{eqn:ladung} die Ladungen der Tropfen in
Tabelle \ref{tab:Ladungen}.
Zur Bestimmung der Elementarladung müssten sehr viele Öltröpfchen mit geringer
Ladung betrachtet werden. Beim Plotten der Ladungen sind dann schmale Bereiche
erkennbar, in denen sich die Punkte häufen. Daraus kann dann bereits durch
hingucken jedem Bereich ein ganzzahliges Vielfaches der Elementarladung
zugeordnet werden. Da es in diesem Versuch nicht möglich ist, derart viele
Öltropfen zu betrachten, wird die Elementarladung anhand des Literaturwertes
auf etwa
\begin{align}
  e_\text{grob} = \SI{1.5(3)e-19}{\coulomb}
\end{align}
abgeschätzt. In Tabelle \ref{tab:Ladungen} sind die ganzzahligen Vielfachen $gV$,
die jedem Öltröpfchen anhand des groben Wertes der Elementarladung zugeordnet
werden, aufgelistet.

\begin{table}[h]
  \centering
  \begin{tabular}{S S @{${}\pm{}$} S S}
    \toprule
    {Nr.} & \multicolumn{2}{c}{$q_\text{unkorr}/\SI{e-19}{\coulomb}$} & {$gV$}\\
    \midrule
    1 & 4.5 & 0.3 & 3 \\
    2 & 9.5 & 1.9 & 6 \\
    3 & 3.4 & 1.2 & 2 \\
    4 & 3.5 & 1.0 & 2 \\
    5 & 8.1 & 0.7 & 5 \\
    6 & 10  & 4   & 6 \\
    7 & 15  & 4   & 9 \\
    \bottomrule
  \end{tabular}
  \caption{Berechnete unkorrigierte Ladungen der Öltröpfchen und zugehörige
  ganzzahlige Vielfache.}
  \label{tab:Ladungen}
\end{table}

\FloatBarrier

Es wird eine lineare Ausgleichsrechnung mit den Ladungen und den
ganzzahligen Vielfachen durchgeführt. Dabei werden die Parameter
\begin{align}
  m_\text{unkorr} & = \num{1.67(6)e-19} \\
  b_\text{unkorr} & = \num{0.21(32)e-19}
\end{align}
berechnet. Aus der Steigung folgt sofort die unkorrigierte Elementarladung
\begin{align}
  e_\text{0,unkorr} = \SI{1.67(6)e-19}{\coulomb}.
\end{align}
In Abbildung \ref{fig:elemunkorr} sind die Ladungen gegen die ganzzahligen
Vielfachen und die Ausgleichsgerade aufgetragen.

\begin{figure}
  \centering
  \includegraphics{build/unkor.pdf}
  \caption{Messwerte und Ausgleichsgerade zur Bestimmung der unkorrigierten Elementarladung.
  Es sind die Ladungen gegen die ganzzahligen Vielfachen aufgetragen.}
  \label{fig:elemunkorr}
\end{figure}

\FloatBarrier

\subsection{Unkorrigierte Avogadrokonstante}

Mit der Faradaykonstante \cite{Faraday}
\begin{align}
  F = \SI{96485.3329(6)}{\coulomb\per\mol}
\end{align}
folgt die unkorrigierte Avogadro-Konstante
\begin{align}
  N_\text{A,unkorr} = \frac{F}{e_\text{0,unkorr}} = \SI{5.77(19)e23}{\per\mol}.
\end{align}

\subsection{Korrigierte Elementarladung}

Mit der Cunningham-Korrektur \eqref{eqn:cunningham} ergeben sich die
korrigierten Ladungen in Tabelle \ref{tab:Ladungenkorr}.

\begin{table}[h]
  \centering
  \begin{tabular}{S S @{${}\pm{}$} S S}
    \toprule
    {Nr.} & \multicolumn{2}{c}{$q_\text{korr}/\SI{e-19}{\coulomb}$} & {$gV$}\\
    \midrule
    1 & 5.4  & 0.4 & 3 \\
    2 & 11.9 & 2.4 & 7 \\
    3 & 5.0  & 1.8 & 3 \\
    4 & 4.4  & 1.3 & 3 \\
    5 & 9.9  & 0.9 & 6 \\
    6 & 12.6 & 5.2 & 8 \\
    7 & 18.9 & 5.1 & 12\\
    \bottomrule
  \end{tabular}
  \caption{Berechnete korrigierte Ladungen der Öltröpfchen und zugehörige
  ganzzahlige Vielfache.}
  \label{tab:Ladungenkorr}
\end{table}

\FloatBarrier

Analog zu \ref{sec:unkoel} wird eine Ausgleichsrechnung durchgeführt.
Die berechneten Parameter lauten
\begin{align}
  m_\text{korr} & = \num{1.56(7)e-19} \\
  b_\text{korr} & = \num{0.4(5)e-19}.
\end{align}
Daraus folgt die korrigierte Elementarladung
\begin{align}
  e_\text{0,korr} = \SI{1.56(7)e-19}{\coulomb}.
\end{align}
In Abbildung \ref{fig:elemunkorr} sind die korrigierten Ladungen gegen die ganzzahligen
Vielfachen und die Ausgleichsgerade aufgetragen.

\begin{figure}
  \centering
  \includegraphics{build/korr.pdf}
  \caption{Messwerte und Ausgleichsgerade zur Bestimmung der unkorrigierten Elementarladung.
  Es sind die Ladungen gegen die ganzzahligen Vielfachen aufgetragen.}
  \label{fig:elemkorr}
\end{figure}

\FloatBarrier

\subsection{Korrigierte Avogadrokonstante}

Aus der korrigierten Elementarladung ergibt sich die korrigierte Avogadro-Konstante
\begin{align}
  N_\text{A,korr} = \frac{F}{e_\text{0,korr}} = \SI{6.20(27)e23}{\per\mol}.
\end{align}
