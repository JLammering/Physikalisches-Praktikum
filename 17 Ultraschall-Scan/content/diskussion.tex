\section{Diskussion}
\label{sec:Diskussion}

\subsection{Untersuchung des Acryl-Blocks mit dem A-Scan}

Die relativen Fehler zwischen den Messwerten für die Positionen und Durchmesser
der Störstellen, aufgenommen mit dem A-Scan \ref{tab:AScanStrecken} und
\ref{tab:AScanDicken}, und den
mit der Schieblehre bestimmten \ref{tab:AbmessungenAcryl} sind in Tabelle
\ref{tab:AScanVGL} aufgezählt. Die Fehler für die Positionen
$\delta s_\text{A}$ sind zum Großteil sehr gering
und werden nicht größer als $\SI{25}{\percent}$. Der A-Scan ist also ein
geeignetes Mittel zur Bestimmung der Position der Störstellen. Bei den
Durchmessern ist der relative Fehler deutlich größer und geht bei $c_1$ sogar
über $\SI{50}{\percent}$. Das liegt unter Anderem daran, dass bei der Aufnahme
der Messwerte das absolute Maximum der Peaks abgelesen wurde. Durch diesen
systematischen Fehler sind alle gemessenen Werte kleiner als die mit der
Schieblehre bestimmten, da nicht exakt die Ränder der Störstellen aufgenommen
wurden.
Problematisch ist der A-Scan, wenn es um die Bestimmung des Durchmessers einer
Störstelle, die sich exakt unter einer anderen befindet, geht. Der Großteil der
Schallwellen wird bereits bei der ersten Störstelle reflektiert, sodass
reflektierte Schallwellen an der unteren Störstelle kaum detektiert werden.
Wird der A-Scan an der anderen Seite des zu untersuchenden Objekts
durchgeführt, so kann zumindest die Position der anderen Störstelle sehr
genau bestimmt werden.

\begin{table}[h]
  \centering
  \begin{tabular}{S S S}
    \toprule
    {Störstelle} & {$\delta s_\text{A}/\si{percent}$} &
    {$\delta d_\text{A}/\si{percent}$} \\
    \midrule
    \text{c1} & 0.7 & 54.4\\
    \text{c2} & 1.1 & 39.3\\
    \text{a1} & 16.1 & 6.7\\
    \text{a2} & 9.0 & 7.2\\
    \text{a3} & 4.4 & 28.6\\
    \text{a4} & 3.1 & 25.5\\
    \text{a5} & 1.8 & 28.6\\
    \text{a6} & 1.8 & 23.2\\
    \text{a7} & 1.7 & \\
    \text{a8} & \text{ } & \text{ } \\
    \text{b} & 22.8 & 8.5\\
    \bottomrule
  \end{tabular}
  \caption{Relative Fehler zwischen $s_\text{A,2}$ und $s_\text{lit}$ und
  $d_\text{A}$ und $d_\text{lit}$.}
  \label{tab:AScanVGL}
\end{table}

\subsection{Untersuchung des Acryl-Blocks mit dem B-Scan}

Die relativen Fehler zwischen den Messwerten für die Positionen und Durchmesser
der Störstellen, aufgenommen mit dem B-Scan \ref{tab:BScanZeitenStreckenDicken},
und den mit der Schieblehre bestimmten \ref{tab:AbmessungenAcryl} sind in
Tabelle \ref{tab:BScanVGL} aufgezählt.
