\section{Auswertung}
\label{sec:Auswertung}

\subsection{Innenwinkel des Prismas}

Zur Überprüfung des Innenwinkels am Prisma werden folgende Werte aufgenommen:
\begin{align*}
  \varphi_\text{l} & = \SI{320.0}{\degree} \\
  \varphi_\text{r} & = \SI{72.3}{\degree}.
\end{align*}
Daraus folgt mit
\begin{align}
  \varphi = \frac{360 + \varphi_\text{r} - \varphi_{l}}{2}
\end{align}
und der Standardabweichung \eqref{eqn:standard}
der Innenwinkel
\begin{align}
  \varphi = \SI{56.15}{\degree}.
\end{align}

\subsection{Dispersionskurve}

Die Messwerte zur Bestimmung der Dispersionskurve und die daraus berechneten
Brechungsindize sind in Tabelle \ref{tab:dispersion}
aufgelistet.

\begin{table}[h]
  \centering
  \begin{tabular}{S S S}
    \toprule
    {$\lambda/\si{\nano\meter}$} & {$\eta/\si{\degree}$} &
    {$n$}\\
    \midrule
    706.5 & 47.2 & 1.667\\
    667.8 & 47.5 & 1.670\\
    587.6 & 48.0 & 1.676\\
    501.6 & 48.9 & 1.686\\
    492.2 & 49.1 & 1.689\\
    471.3 & 49.5 & 1.693\\
    447.1 & 49.9 & 1.698\\
    \bottomrule
  \end{tabular}
  \caption{Messwerte zur Bestimmung der Dispersionskurve und berechnete
  Brechungsindize $n$.}
  \label{tab:dispersion}
\end{table}

\FloatBarrier

Mit den Werten $\frac{1}{\lambda^2}$ und $\lambda^2$ und jeweils $n^2$ wird
eine lineare Regression durchgeführt. Dabei werden die Parameter
\begin{align}
  A_0 & = \num{2.712(3)} & A_2 & = \num{3.38(7)e4} \\
  A_0' & = \num{2.93(1)} & A_2' & = -\num{3.3(3)e-7}
\end{align}
und die summierten Fehlerquadrate
\begin{align}
  s_\text{n}^2 & = \num{3.97e-6} \\
  s_{\text{n}'}^2 & = \num{9.48e-5}
\end{align}
berechnet. Es folgt sofort, dass die erste Regression
\begin{align}
  n^2(\lambda) = A_0 + \frac{A_2}{\lambda^2}
\end{align}
die Messwerte besser beschreibt, da die summierten Fehlerquadrate kleiner sind.
In Abbildung \ref{fig:dispersion} sind Messwerte und Regression aufgetragen.

\begin{figure}
  \centering
  \includegraphics{dispersion.pdf}
  \caption{Ausgleichsfunktion $n(\lambda)$ und Messwerte der Dispersion.
  Es ist das
  Quadrat der Brechungsindices $n$ gegen die Wellenlänge $\lambda$ aufgetragen.}
  \label{fig:dispersion}
\end{figure}

\FloatBarrier

\subsubsection{Abbesche Zahl}

Aus den Regressionsparametern und den Funktionswerten
\begin{align}
  n_\text{C} & = \num{1.671(1)}\\
  n_\text{D} & = \num{1.676(1)}\\
  n_\text{F} & = \num{1.691(1)}\\
\end{align}
ergibt sich mit \eqref{eqn:abbe} die
Abbesche Zahl
\begin{align}
  \nu = \num{35.1(7)}.
\end{align}

\subsubsection{Auflösungsvermögen}

Mit der Ableitung
\begin{align}
  n'(\lambda) = \frac{-A_2}{\lambda^3}\Bigl(A_0 + \frac{A_2}{\lambda^2}\Bigr)^{-\frac{1}{2}}
\end{align}
und Gleichung \eqref{eqn:Aufloesung} folgt das Auflösungsvermögen des
Prismenspektralapparats mit einer Basislänge
\begin{align}
  b = \SI{3}{\centi\meter}
\end{align}
bei $\lambda_\text{C}$
\begin{align}
  A_\text{C} = \num{2.15(4)e-6}
\end{align}
und bei $\lambda_\text{F}$
\begin{align}
  A_\text{F} = \num{5.2(1)e-6}.
\end{align}

\subsubsection{Absorptionsstelle}

Durch einen Koeffizientenvergleich von \eqref{eqn:zwischenformelfuertimo} und
\eqref{eqn:gestalta}
folgt die Formel zur Bestimmung der Absorptionsstelle, die
dem sichtbaren Spektrum am nächsten liegt,
\begin{align}
  \lambda_1 = \sqrt{\frac{A_2}{A_0 - 1}}.
\end{align}
Mit den berechneten Parametern ergibt sich
\begin{align}
  \lambda_1 = \SI{141(2)}{\nano\meter}.
\end{align}
