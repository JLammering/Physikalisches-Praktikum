\section{Diskussion}
\label{sec:Diskussion}

Die summierten Fehlerquadrate bei der benutzten Ausgleichsfunktion sind mit
\begin{align}
  s_\text{n}^2 = \num{3.97e-6}
\end{align}
sehr gering. Das ist auch am Graphen in Abbildung \ref{fig:dispersion} erkennbar.
Die berechnete Abbesche Zahl des Glasprismas
\begin{align}
  \nu = \num{35.1(7)}
\end{align}
liegt zwischen $30$ und $40$ und damit im Bereich starker Farbzerstreuung.
Die Absorptionsstelle, die dem sichtbaren Spektrum am nächsten ist, liegt
bei
\begin{align}
  \lambda_1 = \SI{141(2)}{\nano\meter}
\end{align}
und damit im ultravioletten Bereich.
