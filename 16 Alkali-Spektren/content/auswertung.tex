\section{Auswertung}
\label{sec:Auswertung}

Die genutzten Naturkonstanten \cite{Codata} sind:

\begin{align*}
  h &= \SI{6.626070040(81)e-34}{\joule\second}\\
  c &= \SI{299792458}{\meter\per\second}\\
  e &= \SI{1.6021766208(98)e-19}{\coulomb}\\
  R_{\infty} &= \SI{13.605693009(84)}{\electronvolt}\\
  \alpha &= 7.2973525664(17)e-3
\end{align*}

\subsection{Bestimmung der Gitterkonstante}
\label{sec:gitterkonstante}

In Tabelle \ref{tab:gitterdaten} sind die gemessenen Ablenkwinkel ausgehend
von der 0. Ordnung sowie die zugehörigen Wellenlängen der Spektrallinien eingetragen.
Die Beziehung der beiden Größen ist in Formel \eqref{eqn:Gitter} gezeigt.              %verlinken

In Abbildung \ref{fig:plotgitterkonstante} ist $sin(\phi)$ gegen die Wellenlänge
aufgetragen. Außerdem wurde eine lineare Ausgleichsrechnung gemacht und die Ausgleichsgerade
eingezeichnet.

Die ausgegebene Steigung der Geraden ist
\begin{align*}
  m = \SI{9.78(3)e-4}{\per\nano\meter}.\\
  \intertext{daraus ergibt sich die Gitterkonstante:}\\
  g = \SI{1.022(3)}{\micro\meter}
\end{align*}

\begin{figure}
  \centering
  \includegraphics{plotgitterkonstante.pdf}
  \caption{Plot von $sin(\phi)$ gegen $\lambda$.}
  \label{fig:plotgitterkonstante}
\end{figure}

\begin{table}[h]
  \centering
  \begin{tabular}{S S}
    \toprule
    {$\lambda\:/\: \si{\nano\meter}$} & {$\phi\:/\:\si{\degree}$}\\
    \midrule
    447.1 & 26.3\\
    471.3 & 27.9\\
    492.2 & 29.2\\
    501.6 & 29.8\\
    504.8 & 30.1\\
    587.6 & 35.6\\
    667.8 & 41.3\\
    706.5 & 44.2\\
    \bottomrule
  \end{tabular}
  \caption{Die gegebenen Werte für $\lambda$ und der gemessene Ablenkwinkel
  $\phi$.}
  \label{tab:gitterdaten}
\end{table}

\subsection{Bestimmung der Eichgröße}
\label{sec:eichgroesse}

Die Eichgröße $\psi$ wird zwei Mal anhand unterschiedlicher Spektrallinien
bestimmt.

Die Formel lautet:

\begin{equation}
  \psi = \frac{\lambda_1-\lambda_2}{\Delta t cos(\bar{\phi}_{12}) }
\end{equation}

Die berechneten Eichgrößen und die zugehörigen Messdaten sind in Tabelle
\ref{tab:eichgroesse} zu finden.

\begin{table}[h]
  \centering
  \begin{tabular}{S S S
    S[table-format=1.3]
    @{${}\pm{}$}
    S[table-format=1.3]
    S
    S[table-format=1.3]
    @{${}\pm{}$}
    S[table-format=1.3]}
    \toprule
    {$\lambda_1\:/\: \si{\nano\meter}$} & {$\lambda_2\:/\: \si{\nano\meter}$} &
    {$\lambda_1-\lambda_2\:/\: \si{\nano\meter}$} & \multicolumn{2}{c}{$\bar{\phi}_{12}\:/\:\si{\radian}$}
     & {$\Delta t\:/\:Skt$} & \multicolumn{2}{c}{$\psi\:/\:\si{10^{-11}\meter}Skt^{-1}$}\\
    \midrule
    501.6 & 492.2 & 9.4 & 0.515 & 0.005 & 373 & 2.045 & 0.006\\
    504.8 & 501.6 & 3.2 & 0.523 & 0.003 & 125 & 2.218 & 0.003\\
    \bottomrule
  \end{tabular}
  \caption{Die Eichgöße \psi mit den zugehörigen Messdaten.}
  \label{tab:eichgroesse}
\end{table}

Die gemittelte Eichgröße beträgt:

\begin{align*}
  \psi = \SI{2.131(3)e-11}{\meter}\g{Skt}^{-1}
\end{align*}

\subsection{Abschirmkonstante}

Nun wird im letzten Teil die Abschirmkonstante bestimmt. %Dazu müssen aber zunächst einige andere Werte bestimmt werden.
Aus Formel \eqref{eqn:Gitter} lässt sich mit der Gitterkonstante aus Kapitel
\ref{sec:gitterkonstante} und den gemittelten Winkeln der Dublettlinien die Wellenlänge
$\lambda$ herausfinden. $\increment \lambda$ wird mit Formel \eqref{eqn:dellamda} berechnet,
wobei die Eichgröße aus Kapitel \ref{sec:eichgroesse} eingesetzt wird.
Dann kann mit $\lambda$, $\Delta \lambda$ und Formel \eqref{eqn:delEd} die Energie $\Delta \g{E}_{\g{D}}$
berechnet werden. Eingesetzt in Formel \eqref{eqn:abschirm} ergibt sich dann zusammen mit der passenden Ordnungszahl
$z$ und der Hauptquantenzahl $n$ die Abschirmkonstante $\sigma_2$. Mit $l=1$ gilt:

\begin{equation}
  \sigma_2 = z - \sqrt[4]{2 \Delta\g{E}_{\g{D}} \frac{n^3}{R_{\infty}\alpha^2}}.
  \label{eqn:abschirm}
\end{equation}

In den Tabellen \ref{tab:abschirmna}, \ref{tab:abschirmka} und \ref{tab:abschirmru} sind
die Zwischen- und Endergebnisse zur Berechnung der
Abschirmzahlen $\sigma(n,l)=\sigma(n,1)$ gegeben.

\begin{table}[h]
  \centering
  \begin{tabular}{
    S
    S[table-format=3]
    @{${}\pm{}$}
    S[table-format=1]
    S[table-format=1.4]
    @{${}\pm{}$}
    S[table-format=1.4]
    S[table-format=1.6]
    @{${}\pm{}$}
    S[table-format=1.6]
    S[table-format=1.3]
    @{${}\pm{}$}
    S[table-format=1.3]}
    \toprule
    {$\phi\:/\:\si{\radian}$} & \multicolumn{2}{c}{$\lambda\:/\: \si{\nano\meter}$} & \multicolumn{2}{c}{$\Delta\lambda\:/\:\si{\nano\meter}$}
     & \multicolumn{2}{c}{$\Delta\g{E}_{\g{D}}\:/\:\si{\electronvolt}$} & \multicolumn{2}{c}{$\sigma_2$}\\
    \midrule
    0.656 & 624 & 2 & 0.4559 & 0.0006 & 0.001454 & 0.000009 & 7.774 & 0.004\\
    0.623 & 596 & 2 & 0.5192 & 0.0007 & 0.00181 & 0.00001 & 7.592 & 0.005\\
    0.599 & 576 & 2 & 0.4225 & 0.0006 & 0.00158 & 0.00001 & 7.706 & 0.004\\
    \bottomrule
  \end{tabular}
  \caption{Die Abschirmkonstanten bei Natrium mit $z=11$ und $n=3$.}
  \label{tab:abschirmna}
\end{table}

\begin{table}[h]
  \centering
  \begin{tabular}{
    S[table-format=1.4]
    @{${}\pm{}$}
    S[table-format=1.4]
    S[table-format=3]
    @{${}\pm{}$}
    S[table-format=1]
    S[table-format=1.4]
    @{${}\pm{}$}
    S[table-format=1.4]
    S[table-format=1.6]
    @{${}\pm{}$}
    S[table-format=1.6]
    S[table-format=2.3]
    @{${}\pm{}$}
    S[table-format=1.3]}
    \toprule
    \multicolumn{2}{c}{$\phi\:/\:\si{\radian}$} & \multicolumn{2}{c}{$\lambda\:/\: \si{\nano\meter}$} & \multicolumn{2}{c}{$\Delta\lambda\:/\:\si{\nano\meter}$}
     & \multicolumn{2}{c}{$\Delta\g{E}_{\g{D}}\:/\:\si{\electronvolt}$} & \multicolumn{2}{c}{$\sigma_2$}\\
    \midrule
    0.6082 & 0.0009 & 584 & 2 & 1.486 & 0.002 & 0.00540 & 0.00004 & 13.441 & 0.001\\
    0.6100 & 0.0009 & 585 & 2 & 0.2620 & 0.0004 & 0.000948 & 0.000006 & 15.403 & 0.006\\
    0.5524 & 0.0009 & 536 & 2 & 1.469 & 0.002 & 0.00633 & 0.00004 & 13.216 & 0.001\\
    0.5559 & 0.0009 & 539 & 2 & 1.213 & 0.002 & 0.00517 & 0.00004 & 13.503 & 0.001\\
    \bottomrule
  \end{tabular}
  \caption{Die Abschirmkonstanten bei Kalium mit $z=19$ und $n=4$.}
  \label{tab:abschirmka}
\end{table}

\begin{table}[h]
  \centering
  \begin{tabular}{
    S[table-format=1.3]
    @{${}\pm{}$}
    S[table-format=1.3]
    S[table-format=3]
    @{${}\pm{}$}
    S[table-format=1]
    S[table-format=1.3]
    @{${}\pm{}$}
    S[table-format=1.3]
    S[table-format=1.5]
    @{${}\pm{}$}
    S[table-format=1.5]
    S[table-format=2.2]
    @{${}\pm{}$}
    S[table-format=1.2]}
    \toprule
    \multicolumn{2}{c}{$\phi\:/\:\si{\radian}$} & \multicolumn{2}{c}{$\lambda\:/\: \si{\nano\meter}$} & \multicolumn{2}{c}{$\Delta\lambda\:/\:\si{\nano\meter}$}
     & \multicolumn{2}{c}{$\Delta\g{E}_{\g{D}}\:/\:\si{\electronvolt}$} & \multicolumn{2}{c}{$\sigma_2$}\\
    \midrule
    0.668 & 0.006 & 633 & 5 & 0.452 & 0.005 & 0.00140 & 0.00003 & 32.31 & 0.02\\
    \bottomrule
  \end{tabular}
  \caption{Die Abschirmkonstanten bei Rubidium mit $z=37$ und $n=5$.}
  \label{tab:abschirmru}
\end{table}
