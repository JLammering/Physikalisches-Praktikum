\section{Diskussion}
\label{sec:Diskussion}

Die lineare Ausgleichsrechnung zur Berechnung der Gitterkonstante liegt nah an
den Messwerten.
Zu Erkennen ist dies auch am geringen Fehler der Gitterkonstante. Der Wert liegt
in einer Größenordnung, die bei einem optischen Gitter erwartet wird.

Bemerkenswert ist aber, dass die Winkel nur auf eine Nachkommastelle genau gemessen
werden konnten. Dies wirkt sich besonders auf die Ausmessung der Dublettlinien aus.
Bei der Bestimmung der Eichgröße wurden nur anhand zweier Messungen Werte aufgenommen.
Zur genaueren Bestimmung sind mehrere Durchläufe zu empfehlen.
Die Abschirmkonstanten wurden mit geringem Fehler bestimmt. Zu Bemerken ist, dass
die zweite Zeile bei der Messung von Kalium offensichtliche Messfehler aufweist.
Bei Betrachtung der Messdaten fällt auf, dass der Abstand der Dublettlinien an der Stelle
so nicht stimmen kann. Eine mögliche Fehlerquelle ist hier die teils geringe
Intensität der Spektrallinien
Da die Abschirmung immer noch eine Anziehung des Kerns auf die äußeren Elektronen
zulassen muss, sind die Werte sehr vernünftig.
