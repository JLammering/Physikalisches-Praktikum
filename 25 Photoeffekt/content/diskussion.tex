\section{Diskussion}
\label{sec:Diskussion}

Der Literaturwert für den Quotienten zwischen Planckschem Wirkungsquantum und
der Elementarladung ist
\begin{align}
  \frac{\symup{h}}{\symup{e}_0}_\text{lit} =  \SI{4.14e-15}{\volt\second}.
\end{align}
Das ergibt eine sehr geringe Abweichung von
\begin{align}
  \delta \frac{\symup{h}}{\symup{e}_0} = \SI{5.1}{\percent}
\end{align}
zwischen Mess- und Literaturwert.
Aus beiden Messungen an der orangenen Spektrallinie tritt hervor, dass bei
einer geringen negativen Spannung noch ein positiver Strom fließt. Das liegt
daran, dass die Elektronen auch vor der Bestrahlung eine kinetische Energie
besitzen, sodass sie gegen eine geringe Bremsspannung anlaufen können.
An Abbildung \ref{fig:c} ist erkennbar, dass der Strom sich einem
Sättigungswert $I_\text{S} \approx \SI{6}{\ampere}$ annähert.
Das liegt daran, dass die Anzahl an ausgelösten Elektronen von der
Lichtintensität abhängt. Bei geringen Spannungen steigt die Stromstärke noch an,
da mehr Elektronen zur Anode gelangen, doch bei größeren Spannungen
$U \geq \SI{10}{\volt}$ gelangen nahezu alle ausgelösten Elektronen unabhängig
von der Spannung zur Anode. Der geringe Gegenstrom, der bei höheren
Gegenspannungen detektiert wird, lässt sich dadurch erklären, dass von
der Kathode durch Verdampfung emittierte Elektronen sich an der Anode ablagern
und durch einen Gegenstrom von dort aus erneut zur Kathode beschleunigt werden.
