\section{Auswertung}
\label{sec:Auswertung}

Die Elementarladung $\symup{e}_0$, das Plancksche Wirkungsquantum $\symup{h}$
und die Lichtgeschwindigkeit $c$ werden den Quellen
\cite{elem}, \cite{planck} und \cite{licht} entnommen. Die Wellenlängen zu den
jeweiligen Farben der Spektrallinien werden Abbildung
\ref{fig:linien} \subref{fig:linienhg} entnommen.

\subsection{Photostrom bei Bremsspannung für verschiedenes Licht}

In den Tabellen \ref{tab:orange}, \ref{tab:gruen}, \ref{tab:tuerkis},
\ref{tab:blau} und \ref{tab:violett} sind die Messwerte für Spannung und
Stomstärke bei verschiedenen Lichtwellenlängen abgebildet. Mit der Wurzel der
Stromstärkemesswerten und den Spannungsmesswerten wird eine lineare
Ausgleichsrechnung durchgeführt. Die Messwerte und zugehörigen
Ausgleichsfunktionen sind in den Abbildungen \ref{fig:orange}, \ref{fig:gruen},
\ref{fig:tuerkis}, \ref{fig:blau} und \ref{fig:violett} abgebildet.

\begin{figure}
  \centering
  \includegraphics{build/orange.pdf}
  \caption{Messwerte und Ausgleichsfunktion der Messung bei orangenem Licht,
  $\lambda=\SI{578}{\nano\meter}$. Es ist die Wurzel der Stromstärke gegen
  die Spannung aufgetragen.}
  \label{fig:orange}
\end{figure}

\FloatBarrier

\begin{figure}
  \centering
  \includegraphics{build/gruen.pdf}
  \caption{Messwerte und Ausgleichsfunktion der Messung bei grünem Licht,
  $\lambda=\SI{546}{\nano\meter}$. Es ist die Wurzel der Stromstärke gegen
  die Spannung aufgetragen.}
  \label{fig:gruen}
\end{figure}

\FloatBarrier

\begin{figure}
  \centering
  \includegraphics{build/tuerkis.pdf}
  \caption{Messwerte und Ausgleichsfunktion der Messung bei türkisem Licht,
  $\lambda=\SI{492}{\nano\meter}$. Es ist die Wurzel der Stromstärke gegen
  die Spannung aufgetragen.}
  \label{fig:tuerkis}
\end{figure}

\FloatBarrier

\begin{figure}
  \centering
  \includegraphics{build/blau.pdf}
  \caption{Messwerte und Ausgleichsfunktion der Messung bei violettem Licht,
  $\lambda=\SI{435}{\nano\meter}$. Es ist die Wurzel der Stromstärke gegen
  die Spannung aufgetragen.}
  \label{fig:blau}
\end{figure}

\FloatBarrier

\begin{figure}
  \centering
  \includegraphics{build/violett.pdf}
  \caption{Messwerte und Ausgleichsfunktion der Messung bei violettem Licht,
  $\lambda=\SI{405}{\nano\meter}$. Es ist die Wurzel der Stromstärke gegen
  die Spannung aufgetragen.}
  \label{fig:violett}
\end{figure}

\FloatBarrier

Die berechneten Regressionsparameter sind in Tabelle \ref{tab:regression}
dargestellt.

\begin{table}[h]
  \centering
  \begin{tabular}{S S @{${}\pm{}$} S S @{${}\pm{}$} S}
    \toprule
    {$\lambda/\si{\nano\meter}$} & \multicolumn{2}{c}{$m$} & \multicolumn{2}{c}{$b$}\\
    \midrule
    578 & -0.83 & 0.03 & 0.34 & 0.01 \\
    546 & -0.60 & 0.03 & 0.45 & 0.01 \\
    492 & -2.53 & 0.13 & 0.60 & 0.02 \\
    435 & -0.83 & 0.03 & 1.02 & 0.02 \\
    366 & -1.38 & 0.02 & 1.19 & 0.01 \\
    \bottomrule
  \end{tabular}
  \caption{Berechnete Regressionsparameter für die verschiedenen Wellenlängen.}
  \label{tab:regression}
\end{table}

\FloatBarrier

Die Gegenspannung $U_\text{g}$ folgt unmittelbar aus den jeweiligen
Abzissenabschnitten und die Frequenz $\nu$ des jeweiligen Lichtes mit
\begin{align}
  \nu = \frac{c}{\lambda}.
\end{align}
In Tabelle \ref{tab:Frequenz} sind die jeweiligen Gegenspannungen und
Frequenzen für die verschiedenen Wellenlängen $\lambda$ abgebildet.
Mit diesen Werten wird erneut eine Ausgleichsrechnung durchgeführt.
Ausgleichsgerade und Werte sind in Abbildung \ref{tab:Frequenz}
aufgetragen.

\begin{table}[h]
  \centering
  \begin{tabular}{S S S @{${}\pm{}$} S}
    \toprule
    {$\lambda/\si{\nano\meter}$} & {$\nu/\si{\tera\hertz}$} & \multicolumn{2}{c}{$U/\si{\volt}$}\\
    \midrule
    578 & 518.7 & 0.34 & 0.01 \\
    546 & 549.1 & 0.45 & 0.01 \\
    492 & 609.3 & 0.60 & 0.02 \\
    435 & 689.2 & 1.02 & 0.02 \\
    366 & 819.1 & 1.19 & 0.01 \\
    \bottomrule
  \end{tabular}
  \caption{Zugehörige Frequenzen und Bremsspannungen zu den jeweiligen Wellenlängen.}
  \label{tab:Frequenz}
\end{table}

\begin{figure}
  \centering
  \includegraphics{build/Frequenz.pdf}
  \caption{Graph der Frequenzen und Bremsspannungen und Ausgleichsgerade.}
  \label{fig:Frequenz}
\end{figure}

\FloatBarrier

Die berechneten Regressionsparameter sind
\begin{align}
  m_\text{f} & = \num{3.93(26)e-15} \\
  b_\text{f} & = \num{-1.72(16)}.
\end{align}
Aus der Steigung folgt das Verhältnis
\begin{align}
  \frac{\symup{h}}{\symup{e}_0} = \num{0.39(3)e-15}\si{\volt\second}
\end{align}
und aus dem Betrag des Abzissenabschnittes die Austrittsarbeit
\begin{align}
  A_\text{K} = |b| = \SI{1.72(16)}{\electronvolt}.
\end{align}

\subsection{Photostrom bei einer Wellenlänge}

In den Tabellen \ref{tab:c1} und \ref{tab:c2} sind die Messwerte der
Photostrom-Messung bei orangenem
Licht, $\lambda=\SI{578}{\nano\meter}$, aufgelistet. In Abbildung
\ref{fig:c} sind diese als Graph aufgetragen.

\begin{figure}
  \centering
  \includegraphics{build/c.pdf}
  \caption{Messwerte der Photostrom-Messung bei Gegen- und
  Beschleunigungsspannung.}
  \label{fig:c}
\end{figure}

\FloatBarrier

Bis zur Bremsspannung
\begin{align}
  U_\text{g} \approx \SI{0}{\volt}
\end{align}
ist der Strom etwa bei einem konstanten negativen Wert. Im Mittel beträgt die
Stromstärke
\begin{align}
  \bar{I}_\text{g} = \SI{0.013(7)}{\nano\ampere}.
\end{align}
An dem Graph ist erkennbar, dass die Stromstärke ab der Bremsspannung zunächst
drastisch steigt und sich dann ab etwa $U= \SI{10}{\volt}$ asymptotisch
einem Sättigungsstrom annähert.

\begin{table}[h]
  \centering
  \begin{tabular}{S S}
    \toprule
    {$I/\si{\nano\ampere}$} & {$U/\si{\volt}$}\\
    \midrule
    0.17   & 0.001 \\
    0.125  & 0.050 \\
    0.08   & 0.100 \\
    0.06   & 0.150 \\
    0.02   & 0.200 \\
    0.018  & 0.220 \\
    0.011  & 0.240 \\
    0.008  & 0.260 \\
    0.004  & 0.280 \\
    0.0015 & 0.300 \\
    0.0005 & 0.350 \\
    0.0      & 0.355 \\
    \bottomrule
  \end{tabular}
  \caption{Messwerte der Spannung und Stromstärke bei orangenem Licht,
  $\lambda=\SI{578}{\nano\meter}$.}
  \label{tab:orange}
\end{table}

\begin{table}[h]
  \centering
  \begin{tabular}{S S}
    \toprule
    {$I/\si{\nano\ampere}$} & {$U/\si{\volt}$}\\
    \midrule
    0.0     & 0.462 \\
    0.006 & 0.430 \\
    0.016 & 0.400 \\
    0.012 & 0.350 \\
    0.060 & 0.300 \\
    0.090 & 0.250 \\
    0.150 & 0.200 \\
    0.250 & 0.150 \\
    0.320 & 0.100 \\
    0.480 & 0.050 \\
    0.600 & 0.001 \\
    \bottomrule
  \end{tabular}
  \caption{Messwerte der Spannung und Stromstärke bei grünem Licht,
  $\lambda=\SI{546}{\nano\meter}$.}
  \label{tab:gruen}
\end{table}

\begin{table}[h]
  \centering
  \begin{tabular}{S S}
    \toprule
    {$I/\si{\nano\ampere}$} & {$U/\si{\volt}$}\\
    \midrule
    0.0   & 0.533 \\
    0.002 & 0.500 \\
    0.005 & 0.450 \\
    0.008 & 0.400 \\
    0.012 & 0.350 \\
    0.014 & 0.300 \\
    0.022 & 0.250 \\
    0.024 & 0.200 \\
    0.028 & 0.150 \\
    0.040 & 0.100 \\
    0.046 & 0.050 \\
    0.050 & 0.001 \\
    \bottomrule
  \end{tabular}
  \caption{Messwerte der Spannung und Stromstärke bei türkisem Licht,
  $\lambda=\SI{492}{\nano\meter}$.}
  \label{tab:tuerkis}
\end{table}

\begin{table}[h]
  \centering
  \begin{tabular}{S S}
    \toprule
    {$I/\si{\nano\ampere}$} & {$U/\si{\volt}$}\\
    \midrule
    0     & 1.050 \\
    0.004 & 1.000 \\
    0.026 & 0.900 \\
    0.068 & 0.800 \\
    0.120 & 0.700 \\
    0.225 & 0.600 \\
    0.380 & 0.500 \\
    0.480 & 0.400 \\
    0.700 & 0.300 \\
    0.840 & 0.200 \\
    1.200 & 0.100 \\
    1.750 & 0.001 \\
    \bottomrule
  \end{tabular}
  \caption{Messwerte der Spannung und Stromstärke bei violettem Licht,
  $\lambda=\SI{435}{\nano\meter}$.}
  \label{tab:blau}
\end{table}

\begin{table}[h]
  \centering
  \begin{tabular}{S S}
    \toprule
    {$I/\si{\nano\ampere}$} & {$U/\si{\volt}$}\\
    \midrule
    0     & 1.190 \\
    0.008 & 1.100 \\
    0.024 & 1.000 \\
    0.038 & 0.900 \\
    0.072 & 0.800 \\
    0.120 & 0.700 \\
    0.170 & 0.600 \\
    0.260 & 0.500 \\
    0.360 & 0.400 \\
    0.400 & 0.300 \\
    0.520 & 0.200 \\
    0.600 & 0.100 \\
    0.780 & 0.001 \\
    \bottomrule
  \end{tabular}
  \caption{Messwerte der Spannung und Stromstärke bei violettem Licht,
  $\lambda=\SI{405}{\nano\meter}$.}
  \label{tab:violett}
\end{table}

\begin{table}[h]
  \centering
  \begin{tabular}{S S}
    \toprule
    {$U/\si{\volt}$} & {$I/\si{\nano\ampere}$}\\
    \midrule
    -19  & -0.025\\
    -18  & -0.024\\
    -17  & -0.024\\
    -16  & -0.022\\
    -15  & -0.022\\
    -14  & -0.020\\
    -13  & -0.020\\
    -12  & -0.019\\
    -11  & -0.019\\
    -10  & -0.016\\
    -9   & -0.015\\
    -8   & -0.016\\
    -7   & -0.014\\
    -6   & -0.012\\
    -5   & -0.011\\
    -4   & -0.011\\
    -3   & -0.009\\
    -2   & -0.008\\
    -1.8 & -0.008\\
    -1.6 & -0.008\\
    -1.4 & -0.008\\
    -1.2 & -0.008\\
    -1.0 & -0.007\\
    -0.8 & -0.006\\
    -0.6 & -0.002\\
    -0.4 & -0.002\\
    -0.2 & -0.006\\
    \bottomrule
  \end{tabular}
  \caption{Messwerte der Stromstärke bei Gegenspannungen und der
  Wellenlänge $\lambda=\SI{578}{\nano\meter}$.}
  \label{tab:c1}
\end{table}

\begin{table}[h]
  \centering
  \begin{tabular}{S S}
    \toprule
    {$U/\si{\volt}$} & {$I/\si{\nano\ampere}$}\\
    \midrule
    0    & 0.02\\
    0.2  & 0.12\\
    0.4  & 0.6\\
    0.6  & 0.86\\
    0.8  & 0.95\\
    1.0  & 1.30\\
    1.2  & 1.50\\
    1.4  & 1.60\\
    1.6  & 1.70\\
    1.8  & 1.60\\
    2.0  & 1.80\\
    3    & 2.25\\
    4    & 2.50\\
    5    & 3.80\\
    6    & 4.20\\
    7    & 4.60\\
    8    & 5.00\\
    9    & 4.20\\
    10   & 4.80\\
    11   & 4.60\\
    12   & 5.00\\
    13   & 5.60\\
    14   & 5.40\\
    15   & 4.60\\
    16   & 5.40\\
    17   & 5.10\\
    18   & 4.90\\
    19   & 4.80\\
    \bottomrule
  \end{tabular}
  \caption{Messwerte der Stromstärke bei Beschleunigungsspannungen und der
  Wellenlänge $\lambda=\SI{578}{\nano\meter}$.}
  \label{tab:c2}
\end{table}
