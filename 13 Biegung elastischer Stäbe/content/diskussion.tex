\section{Diskussion}
\label{sec:Diskussion}

\subsection{Runder Stab}

\subsubsection{Bestimmung des Materials anhand der Dichte}

Mit der berechneten Dichte des runden Stabes kann durch einen
Vergleich mit einer Auswahl an Werkstoffen und deren zugehöriger
Dichte \cite{Metallwerte} das benutzte Material bestimmt werden.
Die aus den Messwerten berechete Dichte des runden Stabes beträgt
\begin{equation}
  \rho_\text{r} = \SI{7.79(8)e03}{\kilo\gram\per\cubic\meter}.
\end{equation}
Am nächsten kommt diesem Wert die Dichte von Eisen \cite{Metallwerte}
\begin{equation}
  \rho_\text{Fe} = \SI{7.87e03}{\kilo\gram\per\cubic\meter}.
\end{equation}
Die Abweichung zwischen Mess- und Literaturwert ist mit
\begin{equation}
  1-\frac{\rho_\text{r}}{\rho_\text{Fe}} = \SI{1.02}{\percent}
\end{equation}
gering.

\subsubsection{Elastizitätsmodul}

Das berecheten Elastizitätsmodul für den runden Eisen-Stab beträgt
\begin{equation}
  E_\text{1,r} = \SI{1433(4)e08}{\newton\per\meter\squared}
\end{equation}
und der Literaturwert \cite{Metallwerte}
\begin{equation}
  E_\text{Fe} = \SI{2197e08}{\newton\per\meter\squared}
\end{equation}
Die Abweichung
\begin{equation}
  1-\frac{E_\text{1,r}}{E_\text{Fe}} = \SI{34.79}{\percent}
\end{equation}
ist gering.

\subsection{Quadratischer Stab}

\subsubsection{Bestimmung des Materials anhand der Dichte}

Die aus den Messwerten berechete Dichte des quadratischen Stabes beträgt
\begin{equation}
  \rho_\text{q} = \SI{2.79(1)e03}{\kilo\gram\per\cubic\meter}.
\end{equation}
Am nächsten kommt diesem Wert die Dichte von Aluminium \cite{Metallwerte}
\begin{equation}
  \rho_\text{Al} = \SI{2.70e03}{\kilo\gram\per\cubic\meter}.
\end{equation}
Die Abweichung zwischen Mess- und Literaturwert ist mit
\begin{equation}
  1-\frac{\rho_\text{q}}{\rho_\text{Al}} = -\SI{3.33}{\percent}
\end{equation}
gering.

\subsubsection{Elastizitätsmodul}

Die berechneten Elastizitätsmodule des Aluminium-Stabes betragen
\begin{equation}
  E_\text{1,q} = \SI{640(2)e08}{\newton\per\meter\squared}
\end{equation}
bei der ersten und
\begin{equation}
  E_\text{2,q} = \SI{120(4)e09}{\newton\meter\squared}
\end{equation}
bei der zweiten Messung.
Die Abweichung des Elastizitätsmoduls, das beim einseitig eingespannten Stab
bestimmt wurde, von dem Literaturwert \cite{Metallwerte}
\begin{equation}
  E_\text{Al} = \SI{688e08}{\newton\per\meter\squared}
  \label{eqn:EAlu}
\end{equation}
ist mit
\begin{equation}
  1 - \frac{E_\text{1,q}}{E_\text{Al}} = \SI{6.98}{\percent}
\end{equation}
sehr gering.
Die Abweichung des Elastizitätsmoduls, das beim doppelt aufgelegten Stab
bestimmt wurde, von dem Literaturwert \eqref{eqn:EAlu} ist mit
\begin{equation}
  1 - \frac{E_\text{2,q}}{E_\text{Al}} = -\SI{74.4}{\percent}
  \label{eqn:AbwE2}
\end{equation}
sehr groß. Die Erklärung dafür folgt in Kapitel \ref{sec:SysAbwDoppAuf}.

\subsection{Systematische Abweichungen von den Theoriegleichungen}

\subsubsection{Einseitig eingespannter Stab}

Die Gleichung \eqref{eqn:durchbiegeinseit} für die Durchbiegung des einseitig
eingespannten Stabes in Abhängigkeit von der Entfernung zur eingespannten
Seite beinhaltet die Proportionalität zwischen der Funktion
\begin{equation}
  f_1(x) = Lx^2 - \frac{x^3}{3}
  \label{eqn:f1}
\end{equation}
und $D_1(x)$. In den Abbildungen \ref{fig:Stab1einfachFehler} und
\ref{fig:Stab2einfachFehler} sind jeweils die
Messwerte für die Biegung des einfach eingespannten runden und quadratischen
Stabes $D_1$ gegen die Funktion $f_1(x)$ des gemessenen Abstandes $x_1$
aufgetragen. Zusätzlich werden die Werte jeweils durch eine Ausgleichsgerade
approximiert. Die Abweichung der Steigung
\begin{equation}
  \frac{0.0002}{0.0598} = \SI{0.32}{\percent}
\end{equation}
bei dem runden Stab und
\begin{equation}
  \frac{0.0002}{0.0488} = \SI{0.44}{\percent}
\end{equation}
bei dem quadratischen Stab ist sehr gering, die Theoriegleichung \eqref{eqn:f1}
für einen einfach eingespannten Stab beschreibt das elastische Verhalten
also sehr gut.

\subsubsection{Zweiseitig aufgelegter Stab}
\label{sec:SysAbwDoppAuf}

Aus der Gleichung \eqref{eqn:durchbieg2seit1} folgt die Proportionalität
zwischen
\begin{equation}
  f_2(x) = 3L^2x - 4x^3
  \label{eqn:f2}
\end{equation}
und $D_2(x)$. In Abbildung \ref{fig:Stab2doppeltFehler} sind die
Biegungsmesswerte $D_2$ für die rechte und linke Seite des doppelt
aufgelegten Stabes gegen $f_2(x_2)$ mit den Abstandsmesswerten $x_2$
aufgetragen. Durch die Messwerte für die rechte und linke Seite des Stabes
wird jeweils eine Ausgleichsgerade gelegt.
Die Abweichung der Steigung ist mit
\begin{equation}
  \frac{0.0012}{0.0122} = \SI{10.15}{\percent}
\end{equation}
für die linke Seite und
\begin{equation}
  \frac{0.0007}{0.0107} = \SI{6.79}{\percent}
\end{equation}
für die rechte Seite sehr groß.
Das bedeutet nicht zwangsweise, dass die Theoriegleichung \eqref{eqn:f2}
ungeeignet ist, da es bei der doppelseitigen Auflage des Stabes einige
Fehlerquellen gibt. Mit einem fast vier mal so großen Gewicht wie bei
der einseitigen Einspannung entstehen nur etwa halb so große Biegungen
am Stab. Dadurch haben kleine unerwünschte Verbiegungen am Stab deutlich größere
Auswirkungen auf die Messwerte. Daraus, dass die Messwerte für die linke und
rechte Seite jeweils unterschiedlich verlaufen, lässt sich schließen, dass
der Stab entweder nicht gerade ist, oder das angreifende Gewicht nicht exakt
in der Mitte hängt. Das hat zur Folge, dass das zweite bestimmte
Elastizitätsmodul $E_\text{2,q}$ auch stark von dem Literaturwert abweicht wie
an \eqref{eqn:AbwE2} zu sehen ist.

\begin{figure}
  \centering
  \includegraphics{Stab1einfachFehler.pdf}
  \caption{Biegungsmesswerte $D_\text{1,r}$ des einfach eingespannten, runden
  Stabes gegen $f_1(x_\text{1,r})$ aufgetragen und lineare Regression.}
  \label{fig:Stab1einfachFehler}
\end{figure}

\begin{figure}
  \centering
  \includegraphics{Stab2einfachFehler.pdf}
  \caption{Biegungsmesswerte $D_\text{1,q}$ des einfach eingespannten,
  quadratischen
  Stabes gegen $f_1(x_\text{1,q})$ aufgetragen und lineare Regression.}
  \label{fig:Stab2einfachFehler}
\end{figure}

\begin{figure}
  \centering
  \includegraphics{Stab2doppeltFehler.pdf}
  \caption{Biegungsmesswerte $D_\text{2,q,l}$ und $D_\text{2,q,r}$ des doppelt
  aufgelegten, quadratischen Stabes gegen $f_1(x_\text{2,q})$ aufgetragen und
  lineare Regression für beide Seiten.}
  \label{fig:Stab2doppeltFehler}
\end{figure}
