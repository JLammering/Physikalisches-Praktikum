\section{Diskussion}
\label{sec:Diskussion}

\subsection{Überprüfung der Bragg Bedingung}

Der gemessene Wert zur Überprüfung der Braggbedingung
\begin{align}
  \theta_\text{mess} = \SI{28.4}{\degree}
\end{align}
weicht um
\begin{align}
  \increment \theta = \SI{1.43}{\percent}
\end{align}
vom Sollwert
\begin{align}
  \theta_\text{soll} = \SI{28}{\degree}
\end{align}
ab. Die Braggbedingung wurde also erfolgreich überprüft.

\subsection{Absorptionsspektren verschiedener Stoffe}

Die Mess- und Literaturwerte der Abschirmkonstanten sind in Tabelle
\ref{tab:litwerte} abgebildet.
Die Abweichungen bei Brom und Zirkonium sind mit
\begin{align}
  \increment \sigma_\text{Br} = \SI{19.07}{\percent}
  \increment \sigma_\text{Zr} = \SI{26.58}{\percent}
\end{align}
groß. Das hat unter Anderem mit dem Auflösungsvermögen der Apparatur
\begin{align}
  \increment E = \SI{172(138)}{\electronvolt}
\end{align}
zu tun. Abweichungen in der Größenordnung $10^2$ bei den Energiewerten haben
wegen der komplexen Gleichungen zur Bestimmung der Abschirmkonstanten bereits
ausschlaggebenden Einfluss.

\begin{table}
  \centering
  \begin{tabular}{S S S S S S}
    \toprule
    {Element} & {$Z$} & {$E_\text{K,lit}/\si{\kilo\electronvolt}$} & {$\theta_\text{K,lit}/\si{\degree}$}
    & {$\sigma_\text{K,lit}$} & {$\sigma_\text{K,mess}$} \\
    \midrule
    \text{Ge} & 32 & 11.2036 & 16.1 & 3.6718 & 3.7639 \\
    \text{Br} & 35 & 13.4741 & 13.2 & 3.8429 & 3.2274 \\
    \text{Zr} & 40 & 17.9962 & 9.8 & 4.0950 & 3.2352 \\
    \text{Bi} & 83 & 16.3900 & 10.8 & \text{ } & 0.9265 \\
    \bottomrule
  \end{tabular}
  \caption{Literaturwerte und berechnete Werte für die benutzten Elemente. \cite{NIST}}
  \label{tab:litwerte}
\end{table}

\subsection{Moseleysches Gesetz}

Die aus den Messungen berechnete Rydbergkonstante
\begin{align}
  R_\text{\infty, mess} = \SI{22(2)}{\per\meter}
\end{align}
weicht mit
\begin{align}
  \increment R_\text{\infty} = \SI{61.76}{\percent}
\end{align}
sehr stark vom Literaturwert \cite{anleitung}
\begin{align}
  R_\text{\infty, mess} = \SI{13.6}{\per\meter}
\end{align}
ab. Das lässt sich darauf zurückführen, dass zur Bestimmung der
Ausgleichsgerade, aus deren Steigung die Konstante bestimmt werden kann, nur
drei Messwerte zur Verfügung standen. Kleine Abweichungen bei diesen haben also
sehr große Auswirkungen auf das Ergebnis.
