\section{Diskussion}
\label{sec:Diskussion}

\subsection{Überprüfung der Bragg Bedingung}

Der gemessene Wert zur Überprüfung der Braggbedingung
\begin{align}
  \theta_\text{mess} = \SI{14.2}{\degree}
\end{align}
weicht um
\begin{align}
  \increment \theta = \SI{1.43}{\percent}
\end{align}
vom Sollwert
\begin{align}
  \theta_\text{soll} = \SI{14}{\degree}
\end{align}
ab. Die Braggbedingung wurde also erfolgreich überprüft.

\subsection{Absorptionsspektren verschiedener Stoffe}

Die Mess- und Literaturwerte der Abschirmkonstanten sind in Tabelle
\ref{tab:litwerte} abgebildet.
Die Abweichungen bei Brom und Zirkonium sind mit
\begin{align}
  \increment \sigma_\text{Br} = \SI{19.07}{\percent} \\
  \increment \sigma_\text{Zr} = \SI{26.58}{\percent}
\end{align}
groß. Das hat unter Anderem mit dem Auflösungsvermögen der Apparatur
\begin{align}
  \increment E = \SI{172(138)}{\electronvolt}
\end{align}
zu tun. Abweichungen in der Größenordnung $10^2$ bei den Energiewerten haben
wegen der komplexen Gleichungen zur Bestimmung der Abschirmkonstanten bereits
ausschlaggebenden Einfluss.

\begin{table}
  \centering
  \begin{tabular}{S S S S S S}
    \toprule
    {Element} & {$Z$} & {$E_\text{K,lit}/\si{\kilo\electronvolt}$} & {$\theta_\text{K,lit}/\si{\degree}$}
    & {$\sigma_\text{K,lit}$} & {$\sigma_\text{K,mess}$} \\
    \midrule
    \text{Ge} & 32 & 11.2036 & 16.1 & 3.6718 & 3.7639 \\
    \text{Br} & 35 & 13.4741 & 13.2 & 3.8429 & 3.2274 \\
    \text{Zr} & 40 & 17.9962 & 9.8 & 4.0950 & 3.2352 \\
    \text{Bi} & 83 & 16.3900 & 10.8 & \text{ } & 0.9265 \\
    \bottomrule
  \end{tabular}
  \caption{Literaturwerte und berechnete Werte für die benutzten Elemente. \cite{NIST}}
  \label{tab:litwerte}
\end{table}

\subsection{Moseleysches Gesetz}

Die aus den Messungen berechnete Rydbergkonstante
\begin{align}
  R_\text{\infty, mess} = \SI{22(2)}{\electronvolt}
\end{align}
weicht mit
\begin{align}
  \increment R_\text{\infty} = \SI{61.76}{\percent}
\end{align}
sehr stark vom Literaturwert \cite{anleitung}
\begin{align}
  R_\text{\infty, mess} = \SI{13.6}{\electronvolt}
\end{align}
ab. Das lässt sich darauf zurückführen, dass zur Bestimmung der
Ausgleichsgerade, aus deren Steigung die Konstante bestimmt werden kann, nur
drei Messwerte zur Verfügung standen. Kleine Abweichungen bei diesen haben also
sehr große Auswirkungen auf das Ergebnis.

\section{Anhang}

\begin{table}[h]
  \centering
  \begin{tabular}{S S}
    \toprule
    {$\theta/\si{\degree}$} & {$Imp/\si{\second}$}\\
    \midrule
    26.0 & 54.0\\
    26.1 & 49.0\\
    26.2 & 60.0\\
    26.3 & 65.0\\
    26.4 & 75.0\\
    26.5 & 84.0\\
    26.6 & 99.0\\
    26.7 & 118.0\\
    26.8 & 116.0\\
    26.9 & 131.0\\
    27.0 & 149.0\\
    27.1 & 164.0\\
    27.2 & 165.0\\
    27.3 & 188.0\\
    27.4 & 177.0\\
    27.5 & 199.0\\
    27.6 & 196.0\\
    27.7 & 214.0\\
    27.8 & 202.0\\
    27.9 & 208.0\\
    28.0 & 217.0\\
    28.1 & 225.0\\
    28.2 & 234.0\\
    28.3 & 230.0\\
    28.4 & 237.0\\
    28.5 & 223.0\\
    28.6 & 219.0\\
    28.7 & 219.0\\
    28.8 & 208.0\\
    28.9 & 194.0\\
    29.0 & 186.0\\
    29.1 & 165.0\\
    29.2 & 157.0\\
    29.3 & 135.0\\
    29.4 & 118.0\\
    29.5 & 116.0\\
    29.6 & 96.0\\
    29.7 & 75.0\\
    29.8 & 77.0\\
    29.9 & 64.0\\
    30.0 & 51.0\\
    \bottomrule
  \end{tabular}
  \caption{Messwerte zur Überprüfung der Bragg Bedingung. Es sind die
  Impulse pro Sekunde gegen den Winkel aufgetragen.}
  \label{tab:braggbed}
\end{table}

\begin{table}[h]
  \centering
  \begin{tabular}{S S}
    \toprule
    {$\theta/\si{\degree}$} & {$Imp\si{\second}$}\\
    \midrule
    8.0	& 26.0\\
    8.4	& 26.0\\
    8.8	& 26.0\\
    9.2	& 35.0\\
    9.6	& 46.0\\
    10.0 & 78.0\\
    10.4 & 92.0\\
    10.8 & 118.0\\
    11.2 & 136.0\\
    11.6 & 145.0\\
    12.0 & 182.0\\
    12.4 & 205.0\\
    12.8 & 214.0\\
    13.2 & 223.0\\
    13.6 & 256.0\\
    14.0 & 247.0\\
    14.4 & 269.0\\
    14.8 & 286.0\\
    15.2 & 306.0\\
    15.6 & 316.0\\
    16.0 & 331.0\\
    16.4 & 337.0\\
    16.7 & 353.0\\
    17.2 & 349.0\\
    17.6 & 365.0\\
    18.0 & 367.0\\
    18.4 & 376.0\\
    18.8 & 372.0\\
    \bottomrule
  \end{tabular}
  \caption{Messwerte zur Bestimmung des Emissionsspektrums (1). Es sind die
  Impulse pro Sekunde gegen den Winkel aufgetragen.}
  \label{tab:emission1}
\end{table}

\begin{table}[h]
  \centering
  \begin{tabular}{S S}
    \toprule
    {$\theta/\si{\degree}$} & {$Imp\si{\second}$}\\
    \midrule
    19.2 & 391.0\\
    19.6 & 394.0\\
    20.0 & 402.0\\
    20.4 & 400.0\\
    20.8 & 421.0\\
    21.2 & 389.0\\
    21.6 & 400.0\\
    22.0 & 371.0\\
    22.4 & 369.0\\
    22.8 & 358.0\\
    23.2 & 348.0\\
    23.6 & 342.0\\
    24.0 & 349.0\\
    24.4 & 336.0\\
    24.7 & 320.0\\
    25.2 & 326.0\\
    25.6 & 305.0\\
    26.0 & 274.0\\
    26.4 & 255.0\\
    26.8 & 244.0\\
    27.2 & 229.0\\
    27.6 & 225.0\\
    28.0 & 235.0\\
    28.4 & 218.0\\
    28.8 & 210.0\\
    29.2 & 215.0\\
    29.6 & 208.0\\
    30.0 & 205.0\\
    \bottomrule
  \end{tabular}
  \caption{Messwerte zur Bestimmung des Emissionsspektrums (2). Es sind die
  Impulse pro Sekunde gegen den Winkel aufgetragen.}
  \label{tab:emission2}
\end{table}


\begin{table}[h]
  \centering
  \begin{tabular}{S S}
    \toprule
    {$\theta/\si{\degree}$} & {$Imp\si{\second}$}\\
    \midrule
    30.4 & 202.0\\
    30.8 & 180.0\\
    31.2 & 189.0\\
    31.6 & 176.0\\
    32.0 & 174.0\\
    32.4 & 172.0\\
    32.8 & 176.0\\
    33.2 & 174.0\\
    33.6 & 152.0\\
    34.0 & 158.0\\
    34.4 & 142.0\\
    34.8 & 141.0\\
    35.2 & 153.0\\
    35.5 & 136.0\\
    36.0 & 140.0\\
    36.4 & 135.0\\
    36.8 & 133.0\\
    37.2 & 121.0\\
    37.5 & 119.0\\
    38.0 & 116.0\\
    38.4 & 122.0\\
    38.8 & 143.0\\
    39.2 & 797.0\\
    39.5 & 1196.0\\
    40.0 & 938.0\\
    40.4 & 212.0\\
    40.8 & 158.0\\
    \bottomrule
  \end{tabular}
  \caption{Messwerte zur Bestimmung des Emissionsspektrums (3). Es sind die
  Impulse pro Sekunde gegen den Winkel aufgetragen.}
  \label{tab:emission3}
\end{table}

\begin{table}[h]
  \centering
  \begin{tabular}{S S}
    \toprule
    {$\theta/\si{\degree}$} & {$Imp\si{\second}$}\\
    \midrule
    41.2 & 158.0\\
    41.5 & 142.0\\
    42.0 & 135.0\\
    42.4 & 142.0\\
    42.8 & 139.0\\
    43.2 & 181.0\\
    43.6 & 414.0\\
    44.0 & 3932.0\\
    44.4 & 3527.0\\
    44.8 & 1040.0\\
    45.2 & 160.0\\
    45.5 & 129.0\\
    46.0 & 108.0\\
    46.4 & 106.0\\
    46.8 & 89.0\\
    47.2 & 87.0\\
    47.6 & 87.0\\
    48.0 & 82.0\\
    48.4 & 78.0\\
    48.8 & 75.0\\
    49.2 & 69.0\\
    49.5 & 63.0\\
    50.0 & 62.0\\
    50.4 & 65.0\\
    50.8 & 60.0\\
    51.2 & 64.0\\
    51.6 & 66.0\\
    52.0 & 56.0\\
    \bottomrule
  \end{tabular}
  \caption{Messwerte zur Bestimmung des Emissionsspektrums (4). Es sind die
  Impulse pro Sekunde gegen den Winkel aufgetragen.}
  \label{tab:emission4}
\end{table}

\begin{table}[h]
  \centering
  \begin{tabular}{S S}
    \toprule
    {$\theta/\si{\degree}$} & {$Imp/\si{\second}$}\\
    \midrule
    26.0 & 25.0 \\
    26.2 & 26.0 \\
    26.4 & 26.0 \\
    26.6 & 23.0 \\
    26.8 & 25.0 \\
    27.0 & 24.0 \\
    27.2 & 23.0 \\
    27.4 & 22.0 \\
    27.6 & 22.0 \\
    27.8 & 20.0 \\
    28.0 & 20.0 \\
    28.2 & 20.0 \\
    28.4 & 19.0 \\
    28.6 & 21.0 \\
    28.8 & 21.0 \\
    29.0 & 18.0 \\
    29.2 & 19.0 \\
    29.4 & 18.0 \\
    29.6 & 17.0 \\
    29.8 & 17.0 \\
    30.0 & 19.0 \\
    30.2 & 18.0 \\
    30.4 & 16.0 \\
    30.6 & 18.0 \\
    30.8 & 19.0 \\
    31.0 & 18.0 \\
    31.2 & 23.0 \\
    \bottomrule
  \end{tabular}
  \caption{Messwerte der Germaniumprobe (1). Es sind die
  Impulse pro Sekunde gegen den Winkel aufgetragen.}
  \label{tab:germanium1}
\end{table}


\begin{table}[h]
  \centering
  \begin{tabular}{S S}
    \toprule
    {$\theta/\si{\degree}$} & {$Imp/\si{\second}$}\\
    \midrule
    31.4 & 28.0 \\
    31.6 & 34.0 \\
    31.8 & 40.0 \\
    32.0 & 42.0 \\
    32.2 & 42.0 \\
    32.4 & 39.0 \\
    32.5 & 36.0 \\
    32.8 & 35.0 \\
    33.0 & 37.0 \\
    33.2 & 36.0 \\
    33.4 & 35.0 \\
    33.6 & 32.0 \\
    33.8 & 31.0 \\
    34.0 & 32.0 \\
    34.2 & 29.0 \\
    34.4 & 28.0 \\
    34.5 & 30.0 \\
    34.8 & 29.0 \\
    35.0 & 26.0 \\
    35.2 & 26.0 \\
    35.4 & 26.0 \\
    35.5 & 26.0 \\
    35.8 & 23.0 \\
    36.0 & 22.0 \\
    36.2 & 22.0 \\
    36.4 & 20.0 \\
    36.5 & 20.0 \\
    36.8 & 19.0 \\
    37.0 & 19.0 \\
    37.2 & 19.0 \\
    37.4 & 18.0 \\
    37.5 & 17.0 \\
    37.8 & 17.0 \\
    38.0 & 17.0 \\
    38.2 & 17.0 \\
    38.4 & 15.0 \\
    \bottomrule
  \end{tabular}
  \caption{Messwerte der Germaniumprobe (2). Es sind die
  Impulse pro Sekunde gegen den Winkel aufgetragen.}
  \label{tab:germanium2}
\end{table}

\begin{table}[h]
  \centering
  \begin{tabular}{S S}
    \toprule
    {$\theta/\si{\degree}$} & {$Imp/\si{\second}$}\\
    \midrule
    20.0 & 33.0 \\
    20.2 & 31.0 \\
    20.4 & 33.0 \\
    20.6 & 30.0 \\
    20.8 & 30.0 \\
    21.0 & 29.0 \\
    21.2 & 30.0 \\
    21.4 & 28.0 \\
    21.6 & 29.0 \\
    21.7 & 27.0 \\
    22.0 & 27.0 \\
    22.2 & 26.0 \\
    22.4 & 25.0 \\
    22.6 & 24.0 \\
    22.8 & 25.0 \\
    23.0 & 24.0 \\
    23.2 & 24.0 \\
    23.4 & 23.0 \\
    23.6 & 23.0 \\
    23.8 & 21.0 \\
    24.0 & 23.0 \\
    24.2 & 22.0 \\
    24.4 & 23.0 \\
    24.6 & 22.0 \\
    24.7 & 19.0 \\
    25.0 & 23.0 \\
    25.2 & 28.0 \\
    25.4 & 34.0 \\
    25.6 & 43.0 \\
    25.8 & 47.0 \\
    26.0 & 54.0 \\
    26.2 & 55.0 \\
    26.4 & 53.0 \\
    \bottomrule
  \end{tabular}
  \caption{Messwerte der Bromprobe (1). Es sind die
  Impulse pro Sekunde gegen den Winkel aufgetragen.}
  \label{tab:brom1}
\end{table}

\begin{table}[h]
  \centering
  \begin{tabular}{S S}
    \toprule
    {$\theta/\si{\degree}$} & {$Imp/\si{\second}$}\\
    \midrule
    26.6 & 48.0 \\
    26.8 & 47.0 \\
    27.0 & 46.0 \\
    27.2 & 43.0 \\
    27.4 & 42.0 \\
    27.6 & 41.0 \\
    27.8 & 40.0 \\
    28.0 & 37.0 \\
    28.2 & 34.0 \\
    28.4 & 35.0 \\
    28.6 & 33.0 \\
    28.8 & 32.0 \\
    29.0 & 31.0 \\
    29.2 & 30.0 \\
    29.4 & 29.0 \\
    29.6 & 26.0 \\
    29.8 & 28.0 \\
    30.0 & 27.0 \\
    30.2 & 26.0 \\
    30.4 & 25.0 \\
    30.6 & 23.0 \\
    30.8 & 23.0 \\
    31.0 & 22.0 \\
    31.2 & 22.0 \\
    31.4 & 19.0 \\
    31.6 & 19.0 \\
    31.8 & 21.0 \\
    32.0 & 19.0 \\
    32.2 & 19.0 \\
    32.4 & 16.0 \\
    32.5 & 17.0 \\
    \bottomrule
  \end{tabular}
  \caption{Messwerte der Bromprobe (2). Es sind die
  Impulse pro Sekunde gegen den Winkel aufgetragen.}
  \label{tab:brom2}
\end{table}

\begin{table}[h]
  \centering
  \begin{tabular}{S S}
    \toprule
    {$\theta/\si{\degree}$} & {$Imp/\si{\second}$}\\
    \midrule
    15.0 & 128.0\\
    15.2 & 134.0\\
    15.4 & 138.0\\
    15.6 & 133.0\\
    15.8 & 131.0\\
    16.0 & 136.0\\
    16.2 & 130.0\\
    16.4 & 132.0\\
    16.6 & 131.0\\
    16.7 & 127.0\\
    17.0 & 132.0\\
    17.2 & 128.0\\
    17.4 & 120.0\\
    17.6 & 121.0\\
    17.7 & 121.0\\
    18.0 & 123.0\\
    18.2 & 120.0\\
    18.4 & 132.0\\
    18.6 & 159.0\\
    18.8 & 190.0\\
    19.0 & 229.0\\
    19.2 & 268.0\\
    19.4 & 283.0\\
    19.6 & 294.0\\
    19.8 & 286.0\\
    20.0 & 286.0\\
    20.2 & 285.0\\
    20.4 & 288.0\\
    20.6 & 276.0\\
    20.8 & 280.0\\
    21.0 & 275.0\\
    \bottomrule
  \end{tabular}
  \caption{Messwerte der Zirkoniumprobe (1). Es sind die
  Impulse pro Sekunde gegen den Winkel aufgetragen.}
  \label{tab:zirkonium1}
\end{table}

\begin{table}[h]
  \centering
  \begin{tabular}{S S}
    \toprule
    {$\theta/\si{\degree}$} & {$Imp/\si{\second}$}\\
    \midrule
    21.2 & 279.0\\
    21.4 & 268.0\\
    21.6 & 269.0\\
    21.7 & 260.0\\
    22.0 & 254.0\\
    22.2 & 252.0\\
    22.4 & 242.0\\
    22.6 & 234.0\\
    22.8 & 239.0\\
    23.0 & 223.0\\
    23.2 & 222.0\\
    23.4 & 221.0\\
    23.6 & 213.0\\
    23.8 & 215.0\\
    24.0 & 204.0\\
    24.2 & 205.0\\
    24.4 & 199.0\\
    24.6 & 202.0\\
    24.7 & 197.0\\
    25.0 & 191.0\\
    25.2 & 190.0\\
    25.4 & 178.0\\
    25.6 & 166.0\\
    25.8 & 159.0\\
    26.0 & 155.0\\
    26.2 & 144.0\\
    26.4 & 134.0\\
    26.6 & 130.0\\
    26.8 & 128.0\\
    27.0 & 123.0\\
    \bottomrule
  \end{tabular}
  \caption{Messwerte der Zirkoniumprobe (2). Es sind die
  Impulse pro Sekunde gegen den Winkel aufgetragen.}
  \label{tab:zirkonium2}
\end{table}

\begin{table}[h]
  \centering
  \begin{tabular}{S S}
    \toprule
    {$\theta/\si{\degree}$} & {$Imp/\si{\second}$}\\
    \midrule
    15.8 & 142.0 \\
    16.0 & 147.0 \\
    16.2 & 152.0 \\
    16.4 & 144.0 \\
    16.6 & 149.0 \\
    16.7 & 143.0 \\
    17.0 & 143.0 \\
    17.2 & 145.0 \\
    17.4 & 141.0 \\
    17.6 & 143.0 \\
    17.7 & 140.0 \\
    18.0 & 140.0 \\
    18.2 & 136.0 \\
    18.4 & 130.0 \\
    18.6 & 133.0 \\
    18.8 & 132.0 \\
    19.0 & 125.0 \\
    19.2 & 127.0 \\
    19.4 & 121.0 \\
    19.6 & 121.0 \\
    19.8 & 119.0 \\
    20.0 & 117.0 \\
    20.2 & 111.0 \\
    20.4 & 117.0 \\
    \bottomrule
  \end{tabular}
  \caption{Messwerte der Bismuthprobe (1). Es sind die
  Impulse pro Sekunde gegen den Winkel aufgetragen.}
  \label{tab:bismuth1}
\end{table}

\begin{table}[h]
  \centering
  \begin{tabular}{S S}
    \toprule
    {$\theta/\si{\degree}$} & {$Imp/\si{\second}$}\\
    \midrule
    20.6 & 120.0 \\
    20.8 & 112.0 \\
    21.0 & 118.0 \\
    21.2 & 117.0 \\
    21.4 & 125.0 \\
    21.6 & 128.0 \\
    21.7 & 134.0 \\
    22.0 & 134.0 \\
    22.2 & 0.0 \\
    22.4 & 133.0 \\
    22.6 & 127.0 \\
    22.8 & 128.0 \\
    23.0 & 120.0 \\
    23.2 & 121.0 \\
    23.4 & 113.0 \\
    23.6 & 116.0 \\
    23.8 & 111.0 \\
    24.0 & 108.0 \\
    24.2 & 108.0 \\
    24.4 & 103.0 \\
    24.6 & 101.0 \\
    24.7 & 100.0 \\
    25.0 & 99.0 \\
    25.2 & 98.0 \\
    25.4 & 109.0 \\
    25.6 & 117.0 \\
    25.8 & 123.0 \\
    26.0 & 128.0 \\
    26.2 & 133.0 \\
    26.4 & 128.0 \\
    26.6 & 127.0 \\
    26.8 & 117.0 \\
    27.0 & 118.0 \\
    27.2 & 118.0 \\
    \bottomrule
  \end{tabular}
  \caption{Messwerte der Bismuthprobe (2). Es sind die
  Impulse pro Sekunde gegen den Winkel aufgetragen.}
  \label{tab:bismuth2}
\end{table}
