\section{Theorie}
\label{sec:Theorie}

\subsection{Fehlerrechnung}

Für die Fehlerfortpflanzung bei Gleichungen mit $N$ fehlerbehafteten Größen
wird jeweils die Formel zur Gaußschen Fehlerfortpflanzung

\begin{equation}
  \sigma = \sqrt{\sum_{i=1}^{N}\biggl(\frac{\partial f(x_i)}{\partial x_i}
  \sigma_i\biggr)^2}
\end{equation}
mit der jeweiligen Funktion $f(x_i)$, den Messgrößen $x_i$ und den
zugehörigen Fehlern $\sigma_i$ verwendet.
Zur Berechnung des arithmetischen Mittels von $N$ Messwerten wird jeweils die
Formel

\begin{equation}
  \bar{x} = \frac{1}{N}\sum_{i=1}^{N}x_i
\end{equation}
mit den Messwerten $x_i$ benutzt.
die Standardabweichung des Mittelwerts wird jeweils mit der Gleichung

\begin{equation}
  \bar{\sigma} = \sqrt{\frac{1}{N-1}\sum_{i=1}^{N}(x_i - \bar{x})^2}
\end{equation}
mit den $N$ Messwerten $x_i$ berechnet.

\subsection{Erzeugung von Röntgenstrahlung}

Bei der Erzeugung von Röntgenstrahlung werden Elektronen, die aus einer Glühkathode
austreten durch eine Beschleunigungsspannung innerhalb einer evakuierten Röhre
auf eine Anode  hin beschleunigt. Die dann entstehende Röntgenstrahlung setzt sich
aus \emph{kontinuierlicher Bremsstrahlung} und \emph{charakteristischer Röntgenstrahlung}
zusammen.

\subsubsection{Die kontinuierliche Bremsstrahlung}

Die beschleunigten freien Elektronen werden vom Coulombfeld der Atome des Anodenmaterials
abgebremst; dabei wird ein Photon ausgesendet, das der Bewegungsenergie des  Elektrons
entspricht. Beim Bremsvorgang kann die gesamte oder aber auch nur ein Teil der
kinetischen Energie umgewandelt werden. Deshalb ist das Spektrum kontinuierlich.
Aus der kinetischen Energie 

\begin{align}
  E_{\g{kin}} = \g{e_0} U \label{eqn:Ekin}\\
  E = \g{h} \nu \label{eqn:Estrahl}
\end{align}

\subsubsection{Die charakteristische Röntgenstrahlung}



\cite{anleitung}
