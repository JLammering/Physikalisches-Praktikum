\section{Diskussion}
\label{sec:Diskussion}

\subsection{Einfachspalte}
Der relative Fehler bei der Bestimmung der Spaltbreite mit Hilfe
des Mikroskops ist bei allen drei Spalten verhältnismäßig groß und
wird bei den größeren Spalten größer. Dies liegt daran, dass die Eichung
des Kästchens nicht genau erfolgen kann. Die Vergrößerung des Mikroskops
erfolgt in groben Schritten und stoppt bei vier.

Die Ausgleichsrechnungen der Intensitätsmessungen ergeben Kurven, die die Lage der Messpunkte gut beschreiben.
Auch die, sich aus dem Fit ergebenden, Parameter haben einen sehr kleinen Fehler.
Zu beobachten ist, dass der Fehler bei größerem Spalt größer wird. Das kann so erklärt
werden, dass kleinere Spalte das Licht mehr beugen und deshalb mehr Werte
die von der Theorie beschriebene Beugungsstruktur aufweisen. Wie beispielsweise in Abbildung
\ref{fig:einzelspalt3} zu sehen liegen viele Werte nahe null.

\subsection{Doppelspalt}

Die Messung mit dem Mikroskop ist wieder sehr ungenau.
Die Bestimmung per Intensitätsmessung liefert wieder sehr genaue
Werte.

Der Vergleich der Intensitätsprofile zeigt, dass die Einhüllende der
Doppelspaltfunktion dem Profil eines Einzelspalts gleicht. Außerdem kann aus
Abbildung \ref{fig:vergleich} erkannt werden, dass Spalte mit geringerer Breite ein schmaleres
und höheres Profil aufweisen, da das Licht weniger gebeugt wird
und mehr Licht durchgelassen wird.
