\section{Auswertung}
\label{sec:Auswertung}

\subsection{Schubmodul}

Um den Schubmodul zu bestimmen wird Formel \eqref{eqn:schubG} genutzt.                  %label einfügen
Die Messwerte für $T$ sind in Tabelle \ref{tab:Tvert} aufgetragen.

\begin{table}[h]
  \centering
  \caption{Messwerte der Periodendauern bei vertikalem Magnet in der Kugel.}
  \label{tab:Tvert}
  \begin{tabular}{S}
    \toprule
    {$T\:/\:\si{\second}$} \\
    \midrule
    19.771 \\
    19.826 \\
    19.885 \\
    19.844 \\
    19.805 \\
    19.807 \\
    19.867 \\
    19.842 \\
    19.805 \\
    19.804 \\
    \bottomrule
  \end{tabular}
\end{table}

Der Mittelwert mit Standardabweichung lautet dann:
\begin{equation*}
  T = \SI{19.83 +- 0.03}{\second}.
\end{equation*}

Die Masse der Kugel ist mit
\begin{equation*}
  m_{\text{k}}= \SI{0.5883 +- 0.0002}{\kilogram}
\end{equation*}

angegeben.

Der Radius des Drahts $R$ wird fünf Mal gemessen und dann gemittelt.
Die Messwerte sind in Tabelle \ref{tab:raddraht} zu sehen.

Der Mittelwert mit Standardabweichung lautet:

\begin{equation*}
  R = \SI{0.1007 +- 0.0005}{\milli\meter}
\end{equation*}

\begin{table}[h]
  \centering
  \caption{Messwerte des Radius des Drahts.}
  \label{tab:raddraht}
  \begin{tabular}{S}
    \toprule
    {$R\:/\:\si{\milli\meter}$} \\
    \midrule
    0.101 \\
    0.1005 \\
    0.1015 \\
    0.100 \\
    0.1005 \\
    \bottomrule
  \end{tabular}
\end{table}

Die Länge $L = L_1 + L_2$ ist in Tabelle \ref{tab:laenge} angegeben.

\begin{table}[h]
  \centering
  \caption{Messwerte der Länge des Drahts.}
  \label{tab:laenge}
  \begin{tabular}{S}
    \toprule
    {$L\:/\:\si{\meter}$} \\
    \midrule
    0.657 \\
    0.658 \\
    0.658 \\
    \bottomrule
  \end{tabular}
\end{table}

Gemittelt ergibt sich dann:

\begin{equation*}
  L = \SI{0.6577 +- 0.0005}{\meter}
\end{equation*}

Außerdem ist $R_{\text{k}}$ mit

\begin{equation*}
  R_{\text{k}} = \SI{0.02552 +- 0.00001}{\meter}
\end{equation*}

angegeben.

Dann ergibt sich für den Schubmodul $G$:

\begin{equation*}
  G = \SI{6.3(1)e10}{\newton\per\meter\squared}
\end{equation*}

Die Abweichung vom Literaturwert $G = \SI{8.2e10}{\newton\per\meter\squared}$
beträgt $\SI{23}{\percent}$.

Mit Hilfe des Schubmoduls und der Gleichungen \eqref{eqn:Emu} und
\eqref{eqn:EQ}
können nun auch $Q$ und $\mu$ berechnet werden. Das Elastizitätsmodul ist mit
$E = \SI{21e10}{\newton\per\meter\squared}$ gegeben.

\begin{equation*}
  \mu = \frac{E}{2G}-1 = \num{6.6(1)e21}
\end{equation*}

\begin{equation*}
  Q = \frac{E}{3(1-2\mu)} = \SI{-5.3(1)e-12}{\newton\per\meter\squared}
\end{equation*}

\subsection{Magnetisches Moment}

\begin{table}[h]
  \centering
  \caption{Messwerte der Periodendauern bei unterschiedlich starkem Magnetfeld.}
  \label{tab:tstrom}
  \begin{tabular}{S S S S S}
    \toprule
    {$T_{m,\SI{0.5}{\ampere}}\:/\:\si{\second}$} & {$T_{m,\SI{1}{\ampere}}\:/\:\si{\second}$}
    & {$T_{m,\SI{1.5}{\ampere}}\:/\:\si{\second}$} & {$T_{m,\SI{2}{\ampere}}\:/\:\si{\second}$}
    & {$T_{m,\SI{2.5}{\ampere}}\:/\:\si{\second}$}\\
    \midrule
    13.613 & 10.931 & 9.339 & 8.169 & 7.396\\
    13.626 & 10.845 & 9.276 & 8.295 & 7.598\\
    13.578 & 10.930 & 9.260 & 8.141 & 7.365\\
    13.621 & 10.843 & 9.341 & 8.306 & 7.617\\
    13.585 & 10.933 & 9.331 & 8.161 & 7.391\\
    13.603 & 10.847 & 9.257 & 8.290 & 7.418\\
    13.618 & 10.933 & 9.330 & 8.204 & 7.332\\
    \bottomrule
  \end{tabular}
\end{table}

Die gemessenen Periodendauern bei unterschiedlicher an der Hemholtzspule
anliegenden Stromstärke sind in Tabelle \ref{tab:tstrom} angegeben.

Außerdem wird die magnetische Flussdichte mit Formel \eqref{eqn:helmholtz}
berechnet.

\begin{equation}
  B = \mu_0 \frac{8 I N}{\sqrt{125}R}
  \label{eqn:helmholtz}
\end{equation}

Der Radius der Helmholtzspule ist mit $R = \SI{72e-3}{\meter}$ und die
Windungszahl mit $N = 80$ gegeben. Die berechneten Werte für $B$ sind
in Tabelle \ref{tab:wertestromplot} angegeben. Außerdem sind dort
die Werte für $\frac{1}{T_m^2}$ hierfür wurden die Werte aus Tabelle
\ref{tab:tstrom} gemittelt.

\begin{table}[h]
  \centering
  \caption{In Abbildung \ref{fig:momentplot} eingetragene Werte.}
  \label{tab:wertestromplot}
  \begin{tabular}{S S S}
    \toprule
    {$I \:/\:\si{\ampere}$} & {$B\:/\:\si{\milli\tesla}$}
    & {$\frac{1}{T_{\text{m}}^2}\:/\:\si{10^{-2}\cdot 1 \per\second\squared}$}\\
    \midrule
    0.5 & 0.50 & 0.54\\
    1 & 1.00 & 0.84\\
    1.5 & 1.50 & 1.15\\
    2 & 2.00 & 1.50\\
    2.5 & 2.50 & 1.80\\
    \bottomrule
  \end{tabular}
\end{table}

Diese Werte werden in Abbildung \ref{fig:momentplot} gezeigt und eine Ausgleichsrechnung
nach linearer Regression durchgeführt.

\begin{figure}[h]
  \centering
  \includegraphics[width=\textwidth]{build/momentplot.pdf}
  \caption{Graph zur Bestimmung des magnetischen Moments.}
  \label{fig:momentplot}
\end{figure}

Die Ausgleichsrechnung liefert dann für die Steigung $a$, den Achsenabschnitt $b$
und den Fehler der Regression $\sigma_{\text{a}}$:

\begin{align*}
  a &= \SI{6.33 +- 0.06}{\ampere\per\kilogram} & b &= \SI{2.15e-3}{\frac{1}{\second\squared}}\\
  \sigma_{\text{a}} &= \SI{6e-2}{\ampere\per\kilogram}
\end{align*}

Nun wird von der Steigung auf das magnetische Moment geschlossen.
Zur Bestimmung des magnetischen Moments $m$ wird die Formel \eqref{eqn:altmagmom}      %verlinken!!
umgestellt zu Formel \eqref{eqn:magmom}.

\begin{equation}
  \frac{1}{T_{\text{m}}^2} = \frac{m}{4 \pi^2 \theta} \cdot B + \frac{D}{4 \pi^2 \theta}
  \label{eqn:magmom}
\end{equation}

Dann kann aus der Steigung nach Gleichung \eqref{eqn:steigung} das magnetische Moment
berechnet werden.

\begin{align}
  \intertext{mit}
  \theta &= \theta_{\text{Kugel}} + \theta_{\text{Halterung}} = \SI{0.1555 +- 0.0001}{\cdot10^{-3} \kilogram\meter\squared}\\
  \intertext{ergibt sich}
  m &= a 4 \pi^2 \theta = \SI{0.00971 +- 0.00009}{\ampere\meter\squared}\label{eqn:steigung}.
\end{align}

\FloatBarrier

\subsection{Erdmagnetfeld}

Da das magnetische Moment nun bekannt ist, kann die horizontale Komponente des
Erdmagnetfeld $B$ bestimmmt werden.

In Tabelle \ref{tab:erdmag} sind die Messwerte aufgetragen, die mit ausgeschalteter
Helmholtz-Spule und horizontal ausgerichtetem Magneten in der Kugel aufgenommen wurden.

\begin{table}[h]
  \centering
  \caption{Messwerte der Periodendauern bei horizontalem Magnet in der Kugel.}
  \label{tab:erdmag}
  \begin{tabular}{S}
    \toprule
    {$T\:/\:\si{\second}$} \\
    \midrule
    19.889 \\
    19.889 \\
    19.838 \\
    19.871 \\
    19.877 \\
    19.899 \\
    19.969 \\
    19.965 \\
    19.943 \\
    19.906 \\
    \bottomrule
  \end{tabular}
\end{table}

Gemittelt ergibt sich $T_{\text{m, horizontal}} = \SI{19.90 +- 0.04}{\second}$.
Da dieser Wert eigentlich kleiner sein müsste, da das Erdmagnetfeld im Gegensatz
zum Aufbau für Tabelle \ref{tab:Tvert} die Schwingung dämpft, wird angenommen, dass
der Magnet um $\SI{180}{\degree}$ verdreht ist. Zur Korrektur wird folgende Rechnung durchgeführt:
\begin{equation}
  T_{\text{m, hor.}} = T_{\text{m, vert.}} - (T_{\text{m, hor.}}-T_{\text{m, vert.}}) = \SI{19.75 +- 0.08}{\second}.
\end{equation}


Mit

\begin{equation}
  D = \SI{1.539 +- 0.005}{\newton\meter}
\end{equation}

kann nun $B$ nach Umstellen der Gleichung \eqref{eqn:altmagmom} berechnet werden:

\begin{equation}
  B = \frac{4\pi^2\theta}{m T_{\text{m, hor.}}^2} - \frac{D}{m} = \SI{3.6(9)e-5}{\tesla}.
\end{equation}

Die Abweichung vom angegebenen Literaturwert $B = \SI{3e-5}{\tesla}$ beträgt $\SI{20}{\percent}$.
