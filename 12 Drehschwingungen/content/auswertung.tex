\section{Auswertung}
\label{sec:Auswertung}

\subsection{Schubmodul}

Um den Schubmodul zu bestimmen wird Formel \eqref{eqn:????} genutzt.                  %label einfügen
Die Messwerte für $T$ sind in Tabelle \ref{tab:Tvert} aufgetragen.

\begin{table}[h]
  \centering
  \caption{Messwerte der Periodendauern.}
  \label{tab:Tvert}
  \begin{tabular}{S}
    \toprule
    {$T\:/\:\si{\second}$} \\
    \midrule
    19.771 \\
    19.826 \\
    19.885 \\
    19.844 \\
    19.805 \\
    19.807 \\
    19.867 \\
    19.842 \\
    19.805 \\
    19.804 \\
    \bottomrule
  \end{tabular}
\end{table}

Der Mittelwert mit Standardabweichung lautet dann:
\begin{equation*}
  T = \SI{19.83 +- 0.03}{\second}.
\end{equation*}

Die Masse der Kugel ist mit
\begin{equation*}
  m_{\text{k}}= \SI{0.5883 +- 0.0002}{\kilogram}
\end{equation*}

angegeben.

Der Radius des Drahts $R$ wird fünf Mal gemessen und dann gemittelt.
Die Messwerte sind in Tabelle \ref{tab:raddraht} zu sehen.

Der Mittelwert mit Standardabweichung lautet:

\begin{equation*}
  R = \SI{0.1007 +- 0.0005}{\milli\meter}
\end{equation*}

\begin{table}[h]
  \centering
  \caption{Messwerte des Radius des Drahts.}
  \label{tab:raddraht}
  \begin{tabular}{S}
    \toprule
    {$R\:/\:\si{\milli\meter}$} \\
    \midrule
    0.101 \\
    0.1005 \\
    0.1015 \\
    0.100 \\
    0.1005 \\
    \bottomrule
  \end{tabular}
\end{table}

Die Länge $L = L_1 + L_2$ ist in Tabelle \ref{tab:laenge} angegeben.

\begin{table}[h]
  \centering
  \caption{Messwerte der Länge des Drahts.}
  \label{tab:laenge}
  \begin{tabular}{S}
    \toprule
    {$L\:/\:\si{\meter}$} \\
    \midrule
    0.657 \\
    0.658 \\
    0.658 \\
    \bottomrule
  \end{tabular}
\end{table}

Gemittelt ergibt sich dann:

\begin{equation*}
  L = \SI{0.6577 +- 0.0005}{\meter}
\end{equation*}

Außerdem ist $R_{\text{k}}$ mit

\begin{equation*}
  R_{\text{k}} = \SI{0.02552 +- 0.00001}{\meter}
\end{equation*}

angegeben.

Dann ergibt sich für den Schubmodul $G$

\begin{equation*}
  G = \SI{6.3(1)e10}{\newton\per\meter\squared}
\end{equation*}
