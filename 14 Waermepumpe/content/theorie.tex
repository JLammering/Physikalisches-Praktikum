\section{Theorie}
\label{sec:Theorie}

\subsection{Fehlerrechnung}

Für die Fehlerfortpflanzung bei Gleichungen mit $N$ fehlerbehafteten Größen
wird jeweils die Formel zur Gaußschen Fehlerfortpflanzung

\begin{equation}
  \sigma = \sqrt{\sum_{i=1}^{N}\biggl(\frac{\partial f(x_i)}{\partial x_i}
  \sigma_i\biggr)^2}
\end{equation}
mit der jeweiligen Funktion $f(x_i)$, den Messgrößen $x_i$ und den
zugehörigen Fehlern $\sigma_i$ verwendet.
Zur Berechnung des arithmetischen Mittels von $N$ Messwerten wird jeweils die
Formel

\begin{equation}
  \bar{x} = \frac{1}{N}\sum_{i=1}^{N}x_i
\end{equation}
mit den Messwerten $x_i$ benutzt.
die Standardabweichung des Mittelwerts wird jeweils mit der Gleichung

\begin{equation}
  \bar{\sigma} = \sqrt{\frac{1}{N-1}\sum_{i=1}^{N}(x_i - \bar{x})^2}
\end{equation}
mit den $N$ Messwerten $x_i$ berechnet.


\subsection{Das Prinzip einer Wärmepumpe}

Der erste Hauptsatz der Thermodynamik
besagt im Wesentlichen, dass die Energie in einem
abgeschlossenen System konstant ist. Demnach wäre es theoretisch möglich,
Wärme von einem kälteren in ein wärmeres Reservoir zu übertragen, was
erfahrungsgemäß aber nicht geschieht.
Der zweite Hauptsatz löst dieses Problem auf, denn er besagt, dass diese
Übertragung nur durch Zufuhr von zusätzlicher mechanischer Energie
mittels einer Wärmepumpe möglich ist. In diesem Fall ergibt sich wegen der
Energieerhaltung die Beziehung
\begin{equation}
  Q_1 = Q_2 + A
  \label{eqn:Energieerhaltung}
\end{equation}
mit der vom wärmeren Reservoir aufgenommenen Wärmemenge $Q_1$, der vom kälteren
Reservoir abgegebenen Wärmemenge $Q_2$ und der, von der Wärmepumpe aufgewandten,
mechanischen Arbeit $A$.
Aus den Hauptsätzen folgt außerdem für die zugehörigen Temperaturen $T_1$ und
$T_2$ im Idealfall
\begin{equation}
  \frac{Q_1}{T_1} - \frac{Q_2}{T_2} = 0.
  \label{eqn:TempWaermeIdeal}
\end{equation}
Dabei wird aber davon ausgegangen, dass es sich um einen reversiblen Prozess
handelt. Gleichung \eqref{eqn:TempWaermeIdeal} gilt also nur, wenn die vom
kälteren Reservoir abgegebene Wärme $Q_2$ und die mechanische Arbeit $A$
jederzeit wieder in einem umgekehrten Prozess vollständig zurückgewonnen werden
kann. Da man praktisch nie von einem abgeschlossenen System ausgehen kann, da
beispielsweise immer ein Wärmeaustausch mit der Umgebung stattfindet, muss
Gleichung \eqref{eqn:TempWaermeIdeal} zu
\begin{equation}
  \frac{Q_1}{T_1} - \frac{Q_2}{T_2} > 0
  \label{eqn:TempWaermeReal}
\end{equation}
abgewandelt werden.
Das Verhätnis zwischen der vom wärmeren Reservoir aufgenommenen Wärmemenge $Q_1$
zwischen der geleisteten mechanischen Arbeit bezeichnet man als Güteziffer $\nu$
der Wärmepumpe. Diese gibt an, wie günstig die Wärmepumpe arbeitet, also wie
wenig Arbeitsaufwand für die Übertragung der Wärme notwendig ist.
Mit den Gleichungen \eqref{eqn:Energieerhaltung}, \eqref{eqn:TempWaermeIdeal}
und \eqref{eqn:TempWaermeReal} ergibt sich die Güteziffer einer idealen
Wärmepumpe
\begin{equation}
  \nu_\text{id} = \frac{Q_1}{A} = \frac{T_1}{T_1-T_2}
\end{equation}
und die einer realen Wärmepumpe
\begin{equation}
  \nu_\text{re} < \frac{T_1}{T_1-T_2}.
\end{equation}
An diesen Gleichungen ist erkennbar, dass eine Wärmepumpe umso günstiger
arbeitet, wenn die Temperaturdifferenz $T_1-T_2$ zwischen den beiden Reservoiren
gering ist.
Der Vorteil der Wärmegewinnung mit einer Wärmepumpe gegenüber Verfahren, in
denen mechanische Arbeit direkt in Wärme umgewandelt wird, besteht darin, dass
die gewonnene Wärmemenge $Q_1$ hierbei auch größer als die aufgewandte Arbeit
$A$ sein kann.

\subsection{Die Arbeitsweise einer Wärmepumpe}




\cite{anleitung}
