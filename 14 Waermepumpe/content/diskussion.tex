\section{Diskussion}
\label{sec:Diskussion}

Die Temperaturverläufe konnten gut angenähert werden. Der geringe Koeffizient
$A$ bei beiden Ausgleichskurven zeigt, dass der Anstieg der Temperatur fast
linear zur Zeit verläuft.
Die empirischen Güteziffern weichen stark von den theoretischen Werten einer
idealen Wärmepumpe ab. Es können folgende Gründe angegeben werden. Zum einen arbeitet
der Kompressor nicht adiabatisch, wie es bei den theoretischen Werten angenommen wird.
Dann existieren auch Verluste durch Reibung an den Rohrwänden. Außerdem ist der Versuch
nicht reversibel. Des Weiteren sind auch die Isolierungen nicht perfekt.
Diese Fehlerquellen beeinflussen auch den Massendurchsatz und die mechanische
Kompressorleistung.
