\section{Auswertung}
\label{sec:Auswertung}

\subsection{Fehlerrechnung}

Für die Fehlerfortpflanzung bei Gleichungen mit $N$ fehlerbehafteten Größen
wird jeweils die Formel zur Gaußschen Fehlerfortpflanzung

\begin{equation}
  \sigma = \sqrt{\sum_{i=1}^{N}\biggl(\frac{\partial f(x_i)}{\partial x_i}
  \sigma_i\biggr)^2}
\end{equation}
mit der jeweiligen Funktion $f(x_i)$, den Messgrößen $x_i$ und den
zugehörigen Fehlern $\sigma_i$ verwendet.
Zur Berechnung des arithmetischen Mittels von $N$ Messwerten wird jeweils die
Formel

\begin{equation}
  \bar{x} = \frac{1}{N}\sum_{i=1}^{N}x_i
\end{equation}
mit den Messwerten $x_i$ benutzt.
die Standardabweichung des Mittelwerts wird jeweils mit der Gleichung

\begin{equation}
  \bar{\sigma} = \sqrt{\frac{1}{N-1}\sum_{i=1}^{N}(x_i - \bar{x})^2}
\end{equation}
mit den $N$ Messwerten $x_i$ berechnet.

\subsection{Runder Stab}

\subsubsection{Bestimmung der Dichte und des Flächenträgheitsmoments}

In Tabelle \ref{tab:Dichterund} sind Messwerte für die Länge $l_\text{r}$ und
den Durchmesser $d_\text{r}$ des Stabes mit
kreisförmigem Querschnitt dargestellt.

\begin{table}[H]
  \centering
  \caption{Messwerte für den runden Stab.}
  \label{tab:Dichterund}
  \begin{tabular}{c c}
    \toprule
    $l_\text{r}$/\si{\meter} & $d_\text{r}/\si{\milli\meter}$ \\
    \midrule
    0.580 & 10.00 \\
    0.579 & 10.00 \\
    0.580 & 10.00 \\
    0.579 & 10.00 \\
    0.579 & 10.00 \\
    0.579 & 10.10 \\
    0.579 & 10.14 \\
    0.580 & 10.02 \\
    0.579 & 10.00 \\
    0.579 & 10.02 \\
    \bottomrule
  \end{tabular}
\end{table}

Daraus ergeben sich die Mittelwerte und Standardabweichungen

\begin{align}
  l_\text{r} & = \SI{0.5793(5)}{\meter} & d_\text{r} & = \SI{10.03(5)e-03}{\meter}.
\end{align}
Über das Volumen
\begin{equation}
  V_\text{r} = \frac{\pi d_\text{r}^2}{4} l_\text{r} = \SI{4.58(5)e-05}{\cubic\meter}
\end{equation}
und die Masse

\begin{equation}
  m_\text{r} = \SI{0.3564}{\kilo\gram}
\end{equation}
folgt die Dichte des runden Stabes

\begin{equation}
  \rho_\text{r} = \SI{7.79(8)e03}{\kilo\gram\per\cubic\meter}.
\end{equation}
Aus der Formel \eqref{eqn:momenten} ergibt sich das Flächenträgheitsmoment des runden
Stabes

\begin{equation}
  I_\text{r} = \int_{0}^{2\pi}\int_{0}^{R} r^3 \sin^2(\varphi) \symup{d}r
  \symup{d}\varphi = \frac{\pi R^4}{4} = \frac{\pi d_\text{r}^4}{64} =
  \SI{4.97(10)e-10}{\meter\tothe{4}}.
\end{equation}
\subsubsection{Bestimmung des Elastizitätsmoduls bei einseitiger Einspanung}

In Tabelle \ref{tab:Rundein} sind die Messwerte der Durchbiegungsmessung am
einseitig eingespannten, runden Stab dargestellt.
Dabei ist $x_\text{1,r}$ die Position auf dem Stab, von der eingespannten Seite
aus gesehen, und

\begin{equation}
  D_\text{1,r,a} - D_\text{1,r,b} = D_\text{1,r}
\end{equation}
die Auslenkung des Stabes aus seiner Anfangsposition, in der er nicht belastet
wird. In Tabelle \ref{tab:Rundeinleff} sind die Messwerte für die effektive
Länge des eingespannten runden Stabes aufgeführt.

\begin{table}[H]
  \centering
  \caption{Effektive Länge des einfach eingespannten, runden Stabes.}
  \label{tab:Rundeinleff}
  \begin{tabular}{c}
    \toprule
    $L_\text{eff,1,r}/\si{\meter}$ \\
    \midrule
    0.462 \\
    0.462 \\
    0.462 \\
    \bottomrule
  \end{tabular}
\end{table}

Es ergibt sich der Mittelwert

\begin{equation}
  L_\text{eff,1,r} = \SI{0.462}{\meter}.
\end{equation}
Mit der Masse

\begin{equation}
  M_\text{1,r} = \SI{1.1831}{\kilo\gram}
\end{equation}
des benutzten Gewichts und der Erdbeschleunigung \cite{gWert}

\begin{equation}
  g = \SI{9.81}{\meter\per\second\squared}
  \label{eqn:Erdbeschl}
\end{equation}
folgt die am Stab angreifende Gewichtskraft
\begin{equation}
  F_\text{1,r} = \SI{11.606}{\newton}.
\end{equation}
Mit den berechneten Werten kann die Ausgleichsfunktion
\eqref{eqn:durchbiegeinseit} der Biegung
des runden Stabes und somit der Elastizitätsmodul $E$ bestimmt werden.
In Abbildung \ref{fig:Stab1einfach} sind sowohl die Messwerte, als auch die
Ausgleichsfunktion $D_\text{1,e}(x)$ abgebildet.
Der ausgegebene Funktionsparameter ist der Elastizitätsmodul des runden Stabes

\begin{equation}
  E_\text{1,r} = \SI{1910(5)e08}{\newton\per\meter\squared}.
\end{equation}

\subsection{Quadratischer Stab}

\subsubsection{Bestimmung der Dichte und des Flächenträgheismoments}

In Tabelle \ref{tab:Dichtequad} sind Messwerte für die Länge $l_\text{q}$ und
den Durchmesser $d_\text{q}$ des Stabes mit
quadratischem Querschnitt dargestellt.
Daraus ergeben sich die Mittelwerte und Standardabweichungen

\begin{align}
  l_\text{q} & = \SI{0.5911(3)}{\meter} & d_\text{q} & = \SI{9.97(3)e-03}{\meter}.
\end{align}
Über das Volumen

\begin{equation}
  V_\text{q} = d_\text{q}^2 \cdot l_\text{q} = \SI{5.87(3)e-05}{\cubic\meter}
\end{equation}
und die Masse

\begin{equation}
  m_\text{q} = \SI{0.1641}{\kilo\gram}
\end{equation}
folgt die Dichte des quadratischen Stabes

\begin{equation}
  \rho_\text{q} = \SI{2.79(1)e03}{\kilo\gram\per\cubic\meter}.
\end{equation}

\begin{table}[H]
  \centering
  \caption{Messwerte für den quadratischen Stab.}
  \label{tab:Dichtequad}
  \begin{tabular}{c c}
    \toprule
    $l_\text{q}$/\si{\meter} & $d_\text{q}/\si{\milli\meter}$ \\
    \midrule
    0.591 & 10.00 \\
    0.591 & 9.96 \\
    0.591 & 9.98 \\
    0.591 & 10.00 \\
    0.592 & 9.96 \\
    0.591 & 9.96 \\
    0.591 & 9.92 \\
    0.591 & 9.98 \\
    0.591 & 9.98 \\
    0.591 & 9.94 \\
    \bottomrule
  \end{tabular}
\end{table}

Über die Formel \eqref{eqn:momenten} ergibt sich das Flächenträgheitsmoment des
quadratischen Stabes

\begin{equation}
  I_\text{q} = \int_{-\frac{D}{2}}^{\frac{D}{2}}
  \int_{-\frac{D}{2}}^{\frac{D}{2}} x^2 \symup{d}x \symup{d}y = \frac{D^4}{12} =
  \frac{d_\text{q}^4}{12} = \SI{8.23(8)e-10}{\meter\tothe{4}}.
\end{equation}

\subsubsection{Bestimmung des Elastizitätsmoduls bei einseitiger Einspannung}

In Tabelle \ref{tab:Quadein} sind die Messwerte der Durchbiegungsmessung am
einseitig eingespannten, quadratischen Stab dargestellt.
Dabei ist wieder $x_\text{1,q}$ die Entfernung von der eingespannten Seite
und

\begin{equation}
  D_\text{1,q,a} - D_\text{1,q,b} = D_\text{1,q}
\end{equation}
die Biegung des Stabes.
In Tabelle \ref{tab:Quadeinleff} sind die Messwerte für die effektive
Länge des eingespannten quadratischen Stabes aufgeführt.

\begin{table}[H]
  \centering
  \caption{Effektive Länge des einfach eingespannten, quadratischen Stabes.}
  \label{tab:Quadeinleff}
  \begin{tabular}{c}
    \toprule
    $L_\text{eff,1,q}/\si{\meter}$ \\
    \midrule
    0.482 \\
    0.4815 \\
    0.482 \\
    \bottomrule
  \end{tabular}
\end{table}

Es ergibt sich der Mittelwert

\begin{equation}
  L_\text{eff,1,q} = \SI{0.4818(3)}{\meter}.
\end{equation}
Über die Masse

\begin{equation}
  M_\text{1,q} =  \SI{0.5395}{\kilo\gram}
\end{equation}
des benutzten Gewichts
und die Erdbeschleunigung \eqref{eqn:Erdbeschl}
folgt die am quadratischen Stab angreifende Gewichtskraft

\begin{equation}
  F_\text{1,q} = \SI{5.293}{\newton}.
\end{equation}
Aus diesen Werten wird die Ausgleichsfunktion \eqref{eqn:durchbiegeinseit} und der
Elastizitätsmodul $E$ bestimmt. Die Messwerte und der Fit sind in Abbildung
\ref{fig:Stab2einfach} abgebildet und der Elastizitätsmodul des quadratischen
Stabes lautet

\begin{equation}
  E_\text{1,q} = \SI{640(2)e08}{\newton\per\meter\squared}.
\end{equation}

\subsubsection{Bestimmung des Elastizitätsmoduls bei doppelseitiger Auflage}

In Tabelle \ref{tab:Quadzwei} sind die Messwerte der Durchbiegungsmessung am
doppelt aufgelegten, quadratischen Stab abgebildet. Da das Gewicht bei dieser
Messung in der Mitte hängt, müssen Messwerte für beide Seiten getrennt
aufgenommen werden. Die effektive Länge des aufgelegten Stabes beträgt

\begin{equation}
  L_\text{eff,2,q} = \SI{0.55}{\meter},
\end{equation}
womit der Aufhängepunkt des Gewichtes in der Mitte bei $\SI{0.275}{\meter}$
liegt. Die Werte $x_\text{2,q}$ sind die jeweiligen Abstände der
betrachteten Stellen von dem Aufhängepunkt. Diese werden
für die Ausgleichsrechnungen so umgerechnet, dass sie die Abstände der
der jeweiligen betrachteten Stellen von der linken Auflagestelle beschreiben.
Somit sind die Abstandswerte für die Biegungswerte an der linken Seite

\begin{equation}
  x_\text{2,q,l} = x_\text{2,q},
\end{equation}
mit umgedrehter Zuordnung, und an der rechten Seite

\begin{equation}
  x_\text{2,q,r} = x_\text{2,q} + \SI{0.275}{\meter}.
\end{equation}
Die Biegung des Stabes an der linken Seite ist

\begin{equation}
  D_\text{2,q,l,a} - D_\text{2,q,l,b} = D_\text{2,q,l}
\end{equation}
und die an der rechten Seite

\begin{equation}
  D_\text{2,q,r,a} - D_\text{2,q,r,b} = D_\text{2,q,r}.
\end{equation}
Die Gewichtskraft, die am Stab angreift, ist mit der Masse

\begin{equation}
  M_\text{2,q} = \SI{4.7094}{\kilo\gram}
\end{equation}
und der Erdbeschleunigung \eqref{eqn:Erdbeschl}

\begin{equation}
  F_\text{2,q} = \SI{46.20}{\newton}
\end{equation}
Es werden zwei Ausgleichsfunktionen \eqref{eqn:durchbieg2seit1} und
\eqref{eqn:durchbieg2seit2} und somit
auch zwei Elastizitätsmodule bestimmt.
Die Messwerte für den linken und den rechten Teil des Stabes und die jeweiligen
Fits sind in den Abbildungen \ref{fig:Stab2doppeltlinks} und
\ref{fig:Stab2doppeltrechts} dargestellt.
Aus den daraus resultierenden Elastizitätsmodulen

\begin{equation}
  E_\text{2,q,l} = \SI{109(7)e09}{\newton\meter\squared}
\end{equation}
und

\begin{equation}
  E_\text{2,q,r} = \SI{131(1)e09}{\newton\meter\squared}
\end{equation}
folgt der Mittelwert

\begin{equation}
  E_\text{2,q} = \SI{120(4)e09}{\newton\meter\squared}.
\end{equation}

\begin{figure}
  \centering
  \includegraphics{Stab1einfach.pdf}
  \caption{Messwerte und Fit $D_\text{1,e}(x)$ der Abstandsmessung am einseitig
  eingespannten, runden Stab.}
  \label{fig:Stab1einfach}
\end{figure}

\begin{figure}
  \centering
  \includegraphics{Stab2einfach.pdf}
  \caption{Messwerte und Fit $D_\text{2e}(x)$ der Abstandsmessung am einseitig
  eingespannten, quadratischen Stab.}
  \label{fig:Stab2einfach}
\end{figure}

\begin{figure}
  \centering
  \includegraphics{Stab2doppeltl.pdf}
  \caption{Messwerte und Fit $D_\text{2d,l}(x)$ der Abstandsmessung am
  linken Teil des einseitig eingespannten, quadratischen Stabes.}
  \label{fig:Stab2doppeltlinks}
\end{figure}

\begin{figure}
  \centering
  \includegraphics{Stab2doppeltr.pdf}
  \caption{Messwerte und Fit $D_\text{2d,l}(x)$ der Abstandsmessung am
  rechten Teil des einseitig eingespannten, quadratischen Stabes.}
  \label{fig:Stab2doppeltrechts}
\end{figure}

\begin{table}[h]
  \centering
  \caption{Messwerte für den einfach eingespannten, runden Stab.}
  \label{tab:Rundein}
  \begin{tabular}{c c c}
    \toprule
    $x_\text{1,r}/\si{\meter}$ & $D_\text{1,r,a}/\si{\milli\meter}$ &
    $D_\text{1,r,b}/\si{\milli\meter}$ \\
    \midrule
    0.449 & 9.61 & 5.80 \\
    0.444 & 9.62 & 5.88 \\
    0.439 & 9.62 & 5.95 \\
    0.434 & 9.63 & 6.01 \\
    0.429 & 9.64 & 6.05 \\
    0.424 & 9.64 & 6.14 \\
    0.419 & 9.65 & 6.22 \\
    0.414 & 9.65 & 6.26 \\
    0.409 & 9.66 & 6.35 \\
    0.404 & 9.66 & 6.40 \\
    0.395 & 9.68 & 6.52 \\
    0.385 & 9.69 & 6.67 \\
    0.375 & 9.69 & 6.80 \\
    0.365 & 9.70 & 6.90 \\
    0.355 & 9.70 & 7.03 \\
    0.345 & 9.71 & 7.15 \\
    0.335 & 9.71 & 7.27 \\
    0.325 & 9.70 & 7.38 \\
    0.315 & 9.71 & 7.50 \\
    0.300 & 9.71 & 7.68 \\
    0.280 & 9.72 & 7.89 \\
    0.260 & 9.72 & 8.12 \\
    0.240 & 9.72 & 8.33 \\
    0.220 & 9.72 & 8.52 \\
    0.200 & 9.73 & 8.71 \\
    0.160 & 9.73 & 9.04 \\
    0.120 & 9.71 & 9.30 \\
    0.080 & 9.64 & 9.43 \\
    0.060 & 9.60 & 9.46 \\
    0.030 & 9.54 & 9.49 \\
    \bottomrule
  \end{tabular}
\end{table}

\begin{table}[h]
  \centering
  \caption{Messwerte für den einfach eingespannten, quadratischen Stab.}
  \label{tab:Quadein}
  \begin{tabular}{c c c}
    \toprule
    $x_\text{1,q}/\si{\meter}$ & $D_\text{1,q,a}/\si{\milli\meter}$ &
    $D_\text{1,q,b}/\si{\milli\meter}$ \\
    \midrule
    47.0 & 10.42 & 6.87 \\
    46.5 & 10.42 & 6.92 \\
    46.0 & 10.46 & 7.01 \\
    45.5 & 10.45 & 7.05 \\
    45.0 & 10.45 & 7.12 \\
    44.5 & 10.45 & 7.18 \\
    44.0 & 10.45 & 7.23 \\
    43.5 & 10.54 & 7.36 \\
    42.0 & 10.51 & 7.51 \\
    41.0 & 10.52 & 7.60 \\
    40.0 & 10.50 & 7.68 \\
    39.0 & 10.50 & 7.79 \\
    38.0 & 10.49 & 7.89 \\
    37.0 & 10.48 & 7.98 \\
    36.0 & 10.47 & 8.07 \\
    35.0 & 10.46 & 8.17 \\
    34.0 & 10.45 & 8.28 \\
    33.0 & 10.43 & 8.35 \\
    32.0 & 10.41 & 8.44 \\
    30.0 & 10.38 & 8.61 \\
    28.0 & 10.35 & 8.75 \\
    26.0 & 10.29 & 8.89 \\
    24.0 & 10.25 & 9.03 \\
    22.0 & 10.21 & 9.15 \\
    20.0 & 10.13 & 9.23 \\
    17.0 & 10.05 & 9.37 \\
    14.0 & 9.95 & 9.46 \\
    11.0 & 9.85 & 9.53 \\
    8.0 & 9.71 & 9.54 \\
    3.0 & 9.50 & 9.46 \\
    \bottomrule
  \end{tabular}
\end{table}

\begin{table}[h]
  \centering
  \caption{Messwerte für den doppelt aufgelegten, quadratischen Stab.}
  \label{tab:Quadzwei}
  \begin{tabular}{c c c c c}
    \toprule
    $x_\text{2,q}/\si{\centi\meter}$ & $D_\text{2,q,l,a}/\si{\milli\meter}$ &
    $D_\text{2,q,r,a}/\si{\milli\meter}$ & $D_\text{2,q,l,b}/\si{\milli\meter}$
    & $D_\text{2,q,r,b}/\si{\milli\meter}$\\
    \midrule
    1.0 & 8.49 & 9.03 & 7.07 & 7.65 \\
    2.0 & 8.47 & 9.03 & 7.04 & 7.68 \\
    3.0 & 8.37 & 9.04 & 7.00 & 7.71 \\
    4.0 & 8.40 & 9.06 & 6.97 & 7.78 \\
    5.0 & 8.38 & 9.08 & 6.96 & 7.84 \\
    6.0 & 8.37 & 9.10 & 6.97 & 7.91 \\
    7.0 & 8.37 & 9.12 & 7.00 & 8.00 \\
    8.0 & 8.35 & 9.15 & 7.01 & 8.08 \\
    9.0 & 8.33 & 9.17 & 7.03 & 8.16 \\
    10.0 & 8.33 & 9.19 & 7.08 & 8.24 \\
    12.5 & 8.27 & 9.25 & 7.13 & 8.46 \\
    15.5 & 8.20 & 9.32 & 7.28 & 8.73 \\
    18.5 & 8.15 & 9.35 & 7.39 & 8.95 \\
    21.5 & 8.11 & 9.36 & 7.58 & 9.12 \\
    24.5 & 8.08 & 9.37 & 7.79 & 9.27 \\
    \bottomrule
  \end{tabular}
\end{table}
