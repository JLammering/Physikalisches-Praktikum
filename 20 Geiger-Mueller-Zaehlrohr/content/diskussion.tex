\section{Diskussion}
\label{sec:Diskussion}

Die Charakteristik ist in Abbildung \ref{fig:rohrcharakteristik} gut zu erkennen.
Auch der Übergang zu den Phasen links und rechts vom Plateau lässt sich erahnen.
Die Ausgleichsgerade läuft weitestgehend durch die Fehlerbalken der Messwerte, approximiert
das Plateau also gut.

Die Untersuchung des zeitlichen Abstands zwischen Primär- und Nachentladungsimpulsen
kann nur sehr ungenau erfolgen, da viele Nachentladungen in sehr vielen verschiedenen
Zeitabständen auf dem Oszilloskop auftreten. Deshalb ist der Wert auch mit einem hohen Fehler
versehen.

Die Totzeit konnte in beiden Verfahren auf einen ähnlichen Wert bestimmt werden.
Der Fehler des Werts bei der Zwei-Quellen-Methode legt die Vermutung nahe, dass
dieses Verfahren ungenauer ist.

Die Bestimmung der freigesetzten Ladungsmenge zeigt bei höherer Spannung eine höhere
pro einfallendem Teilchen freigesetzte Ladungsmenge. Dies lässt sich mit der höheren
Beschleunigung der ionisierten Teilchen und den vermehrt auftretenden Nachentladungen
erklären. In Abbildung \ref{fig:ladungsmenge} ist ein linearer Zusammenhang zwischen Spannung
und Ladungsmenge zu erkennen.
