\section{Auswertung}
\label{sec:Auswertung}

\subsection{Aufnahme der Charakteristik des Zählrohrs}

\begin{table}[h]
  \centering
  \begin{tabular}{S
    S[table-format=3.1]
    @{${}\pm{}$}
    S[table-format=1.1]}
    \toprule
    {$U\:/\:\si{\volt}$} & \multicolumn{2}{c}{$N\:/\:\si{\per\second}$}\\
    \midrule
    300 & 0 & 0\\
    320 & 0 & 0\\
    330 & 456.8 & 6.8\\
    340 & 458.1 &  6.8\\
    350 & 465.9 & 6.8\\
    360 & 480.4 & 6.9\\
    380 & 475.1 & 6.9\\
    400 & 472.1 & 6.9\\
    420 & 477.5 & 6.9\\
    440 & 493.6 & 7.0\\
    460 & 482.9 & 6.9\\
    480 & 481.7 & 6.9\\
    500 & 481.6 & 6.9\\
    520 & 490.7 & 7.0\\
    540 & 480.2 & 6.9\\
    560 & 486.2 & 7.0\\
    580 & 491.0 & 7.0\\
    600 & 496.5 & 7.0\\
    620 & 499.3 & 7.1\\
    640 & 505.3 & 7.1\\
    650 & 492.4 & 7.0\\
    660 & 508.7 & 7.1\\
    670 & 512.6 & 7.2\\
    680 & 521.9 & 7.2\\
    700 & 520.8 & 7.2\\
    720 & 567.0 & 7.5\\
    \bottomrule
  \end{tabular}
  \caption{Messwerte zur Charakteristik des Zählrohrs.}
  \label{tab:charakteristik}
\end{table}

In Tabelle \ref{tab:charakteristik} sind die Messwerte zu sehen. Der Fehler wird mit Formel
\eqref{eqn:fehlerN} bestimmt. Die Meßzeit beträgt bei dieser Messung $\SI{10}{\second}$.
\begin{equation}
  \Delta N = \frac{\sqrt{\text{Zählungen}}}{\text{Meßzeit}}
  \label{eqn:fehlerN}
\end{equation}
Die Zählungen sind hierbei die aufgenommenen Impulse in der jeweiligen Meßzeit.

\begin{figure}
  \centering
  \includegraphics[width = \textwidth]{build/rohrcharakteristik.pdf}
  \caption{Die aufgenommene Rohrcharakteristik.}
  \label{fig:rohrcharakteristik}
\end{figure}

In Abbildung \ref{fig:rohrcharakteristik} ist die aufgenommene Charakteristik des Zählrohrs abgebildet.
Die ersten beiden Messwerte aus Tabelle \ref{tab:charakteristik} sind nicht eingetragen, da sie
den Rest des Graphen schwieriger erkennen lassen würden. Zur Ausgleichsrechnung wurden die ersten und
letzten drei Werte des Graphens nicht hinzugezogen, da sie nicht auf dem Plateau liegen.
Die Ausgleichsrechnung ergibt eine Plateausteigung $m$ von:
\begin{align*}
  m = \SI{0.10+-0.02}{\per\volt}.\\
  \intertext{Mit $m$ und folgender Formel ergibt sich die Steigung in $\si{\percent}$:}%steigung in %
  m_{\g{in}\si{\percent}} = \frac{m \cdot \SI{100}{\volt}}{N_{\g{bei}~\SI{500}{\volt}}} = \SI{2.0+-0.3}{\percent}.
\end{align*}

\subsection{Untersuchung der Nachentladungsimpulse}
\label{sec:nachentladung}

Der zeitliche Abstand zwischen Primär- und Nachentladungsimpulsen $T_\text{L}$ kann am Oszilloskop
abgelesen werden. Die Ablenkgeschwindigkeit wird
\begin{equation*}
  \Delta t = \SI{50}{\micro\second\text{DIV}}^{-1}
\end{equation*}
gewählt.
Gemessen werden
\begin{equation*}
  T_\text{L} = \SI{5+-2}{\text{DIV}} = \SI{250+-100}{\micro\second}.
\end{equation*}

\subsection{Bestimmung der Totzeit}

\subsubsection{Oszillographische Messung}

Gemessen wird die Breite des Primärimpulses. Die Ablenkgeschwindigkeit bleibt diejenige aus
Kapitel \ref{sec:nachentladung}.
So ergibt sich eine Totzeit von:

\begin{equation*}
  T = \SI{2.2+-0.1}{\text{DIV}} = \SI{110+-5}{\micro\second}.
\end{equation*}

Außerdem beträgt die Erholungszeit $T_\text{E}$:

\begin{equation*}
  T_\text{E} = \SI{10+-2}{\text{DIV}} = \SI{500+-100}{\micro\second}.
\end{equation*}

\subsubsection{Zwei-Quellen-Methode}

Um eine möglichst genaue Messung zu gewährleisten, werden die Impulse $\SI{60}{\second}$ lang
aufgenommen. Die Messwerte lauten:

\begin{align*}
  N_1 &= \num{401+-3} & N_2 &= \num{361+-3} & N_{1+2} &= \num{742+-4}.
\end{align*}

Die Fehler wurden erneut mit Formel \eqref{eqn:fehlerN} bestimmt. Mit Formel \eqref{eqn:totzeit2quellen}%Referenz Totzeit 2Quellen
~kann dann die Totzeit bestimmt werden. Das Ergebnis ist dann:

\begin{equation*}
  T = \SI{74+-18}{\micro\second}.
\end{equation*}

\subsection{Freigesetzte Ladungsmenge}

Mit Formel \eqref{eqn:ladungproteilchen}%ladungsmenge
berechnet sich die pro einfallendem Teilchen freigesetzte Ladungsmenge.
In Tabelle \ref{tab:ladungsmenge} sind die Messdaten und Ergebnisse zu sehen.

Die genutzte Naturkonstante $\si{\elementarycharge}$ beträgt \cite{codata}:
\begin{equation}
  \si{\elementarycharge} =  \SI{1.6021766208(98)e-19}{\coulomb}.
\end{equation}

\begin{table}[h]
  \centering
  \begin{tabular}{S
    S[table-format=3.1]
    @{${}\pm{}$}
    S[table-format=1.1]
    S
    S[table-format=2.1]
    @{${}\pm{}$}
    S[table-format=1.1]}
    \toprule
    {$U\:/\:\si{\volt}$} & \multicolumn{2}{c}{$N\:/\:\si{\per\second}$} & {$I\:/\:\si{\micro\ampere}$}
    & \multicolumn{2}{c}{$\Delta Q\:/\: 10^9\si{\elementarycharge}$}\\%EINHEIT LADUNG
    \midrule
    300 & 0 & 0 & 0.00 & 0 & 0\\
    320 & 0 & 0 & 0.00 & 0 & 0\\
    330 & 456.8 & 6.8 & 0.20 & 2.7 & 0.7\\
    340 & 458.1 &  6.8 & 0.40 & 5.4 & 0.7\\
    350 & 465.9 & 6.8 & 0.40 & 5.4 & 0.7\\
    360 & 480.4 & 6.9 & 0.40 & 5.2 & 0.7\\
    380 & 475.1 & 6.9 & 0.60 & 7.9 & 0.7\\
    400 & 472.1 & 6.9 & 0.70 & 9.3 & 0.7\\
    420 & 477.5 & 6.9 & 0.80 & 10.5 & 0.7\\
    440 & 493.6 & 7.0 & 0.95 & 12.0 & 0.7\\
    460 & 482.9 & 6.9 & 1.00 & 13.0 & 0.7\\
    480 & 481.7 & 6.9 & 1.10 & 14.3 & 0.7\\
    500 & 481.6 & 6.9 & 1.30 & 16.8 & 0.7\\
    520 & 490.7 & 7.0 & 1.40 & 17.8 & 0.7\\
    540 & 480.2 & 6.9 & 1.50 & 19.5 & 0.7\\
    560 & 486.2 & 7.0 & 1.60 & 20.5 & 0.7\\
    580 & 491.0 & 7.0 & 1.80 & 22.9 & 0.7\\
    600 & 496.5 & 7.0 & 1.90 & 23.9 & 0.7\\
    620 & 499.3 & 7.1 & 2.05 & 25.6 & 0.7\\
    640 & 505.3 & 7.1 & 2.20 & 27.1 & 0.7\\
    650 & 492.4 & 7.0 & 2.20 & 27.9 & 0.7\\
    660 & 508.7 & 7.1 & 2.40 & 29.4 & 0.7\\
    670 & 512.6 & 7.2 & 2.30 & 28.0 & 0.7\\
    680 & 521.9 & 7.2 & 2.50 & 30.0 & 0.7\\
    700 & 520.8 & 7.2 & 2.60 & 31.2 & 0.7\\
    720 & 567.0 & 7.5 & 2.80 & 30.8 & 0.7\\
    \bottomrule
  \end{tabular}
  \caption{Messwerte zur Bestimmung der freigesetzten Ladungsmenge mit $\Delta I = \SI{0.05}{\micro\ampere}$.}
  \label{tab:ladungsmenge}
\end{table}

\begin{figure}
  \centering
  \includegraphics[width = \textwidth]{build/ladungsmenge.pdf}
  \caption{Die Ladungsmenge in Abhängigkeit von der Spannung.}
  \label{fig:ladungsmenge}
\end{figure}

In Abbildung \ref{fig:ladungsmenge} ist die Ladungsmenge gegen die Spannung aus Tabelle
\ref{tab:ladungsmenge} aufgetragen.
