\section{Diskussion}
\label{sec:Diskussion}

\subsection{Halbwertszeiten}

Die Literaturwerte der Halbwertszeiten
\begin{align}
  T_{\frac{1}{2},\text{Iod,lit}} & = \SI{1500}{\second} \\
  T_{\frac{1}{2},\text{Rh,K,lit}} & = \SI{42}{\second} \\
  T_{\frac{1}{2},\text{Rh,L,lit}} & = \SI{264}{\second}
\end{align}
werden der Nuklidkarte \cite{nuklid} entnommen.
Die Abweichungen der Messwerte sind mit
\begin{align}
  \delta T_{\frac{1}{2},\text{Iod}} & = \SI{4.1}{\percent} \\
  \delta T_{\frac{1}{2},\text{Rh,K}} & = \SI{9.5}{\percent} \\
  \delta T_{\frac{1}{2},\text{Rh,L}} & = \SI{21.2}{\percent}
\end{align}
gering und mit den statistischen Abweichungen liegen die Literaturwerten
für jede einzelne Halbwertszeit zwischen den Schranken.

\subsection{Zerfallskurven}

An Abbildung \ref{fig:RhodiumE} ist erkennbar, dass die summierte
Ausgleichsfunktion des Rhodiumisotops die Messung sehr gut beschreibt.
Die Ungleichung der beiden berechneten Exponentialfunktionen
\begin{align}
  N_{\increment t,\text{R,K}} (t^*) < N_{\increment t,\text{R,L}} (t^*)
\end{align}
ist erfüllt.
