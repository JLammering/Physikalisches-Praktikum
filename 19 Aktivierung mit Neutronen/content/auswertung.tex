\section{Auswertung}
\label{sec:Auswertung}

Für den Nulleffekt werden folgende Messwerte für die vom Zählrohr detektierten
Impulse in einer Zeitspanne von
jeweils 900 Sekunden aufgenommen:
\begin{align*}
  P_\text{offset} = [195,202].
\end{align*}
Daraus ergibt sich der Mittelwert und die Standardabweichung vom Nulleffekt
\begin{align*}
  E_\text{offset} = \SI{0.221(6)}{\per\second}.
\end{align*}

\subsection{Zerfallskurve und Halbwertszeit eines Iod-Isotops}

Die Messwerte zur Bestimmung der Zerfallskurve und Halbwertszeit von
$\ce{^{128}_{53}I}$ sind in Tabelle \ref{tab:Iod} dargestellt.
In Abbildung \ref{fig:Iod} sind die Impulsmesswerte gegen die Zeit und die
zugehörige Exponentialfunktion in einem halblogarithmischen Diagramm aufgetragen.
Dabei wird der Nullwert für ein 200-Sekunden-Intervall
\begin{align}
  E_{\text{offset},200} = \SI{44(1)}{\per\second}
\end{align}
bereits von
den Impulsmesswerten abgezogen. Es werden keine Fehlerbalken eingetragen,
da die relativen Abweichungen durch den fehlerbehafteten Offset sehr gering sind.
Bei der linearen Regression der logarithmierten Messdaten werden folgende Werte
für Steigung $m_\text{I}$ und Verschiebung $b_\text{I}$ berechnet:
\begin{align*}
  m_\text{I} & = -\num{4.8(4)e-4}\\
  b_\text{I} & = \num{6.28(6)}\\
\end{align*}
Aus der Steigung $m_\text{I}$ und der Verschiebung $b_\text{I}$ ergibt sich der Faktor
\begin{align}
  N_\text{I}(0)(1-\symup{e}^{-\increment t \lambda_\text{I}}) = \symup{e}^{b_\text{I}} (1-\symup{e}^{200 {m_\text{I}}})
  = \num{49(3)}
\end{align}
und mit \eqref{eqn:HalbwertszeitSteigung} die Halbwertszeit des Iod-Isotops
\begin{align*}
  T_{\frac{1}{2},\text{Iod}} = \SI{1441(110)}{\second}.
\end{align*}

\begin{figure}
  \centering
  \includegraphics{build/Jod.pdf}
  \caption{Messwerte der Iodmessung und Ausgleichsfunktion. Es sind
  die Impulse halblogarithmisch gegen die Zeit aufgetragen.}
  \label{fig:Iod}
\end{figure}

\subsection{Zerfallskurve und Halbwertszeit eines Rhodium-Isotops}

In den Tabellen \ref{tab:Rhodium} und \ref{tab:Rhodium2} sind die Messwerte
zur Bestimmung der Halbwertszeit von Rhodium abgebildet.
Diese sind in Abbildung \ref{fig:RhodiumM} als halblogarithmischer Graph
aufgetragen, wobei der Nullwert für ein 20-Sekunden-Intervall
\begin{align}
  E_\text{offset,20} = \SI{4.4(1)}{\per\second}
\end{align}
bereits abgezogen wird.
Bei
\begin{align}
  t^* = \SI{400}{\second}
\end{align}
wird das erste Isotop $\ce{^{104}_{45}Rh}$ als vollständig zerfallen angenommen, da die
Messwerte etwa ab dieser Stelle eine flachere Gerade bilden.
Mit den logarithmierten Messpunkten für $t_i \geq t^*$ wird eine
lineare Ausgleichsrechnung durchgeführt.
Der Graph dafür ist in einem halblogarithmischen Diagramm in Abbildung
\ref{fig:Rhodium2} dargestellt und die Regressionsparameter lauten
\begin{align}
  m_\text{R,L} & = -\num{2.1(9)e-3} \\
  b_\text{R,L} & = \num{4.1(5)}.
\end{align}
Daraus ergeben sich die Einträge für die Funktionsgleichung des angeregten
Rhodiumisotops
\begin{align}
  N_{\increment t,\text{R,L}} (t) = N_\text{0,R,L}(1-
  \symup{e}^{-\lambda_\text{R,L}\increment t})\symup{e}^{-\lambda_\text{R,L}t}
  \label{eqn:RhodiumL}
\end{align}
mit
\begin{align}
  N_\text{0,R,L}(1-\symup{e}^{-\lambda_\text{R,L}\increment t}) =
  \symup{e}^{b_\text{R,L}} = \num{61(30)}
\end{align}
und
\begin{align}
  \lambda_\text{R,L} = -m_\text{R,L} = \num{2.1(9)e-3}.
\end{align}
Mit \eqref{eqn:HalbwertszeitSteigung} folgt die Halbwertszeit von $\ce{^{104i}_{45}Rh}$
\begin{align}
  T_{\frac{1}{2},\text{Rh,L}} = \SI{320(130)}{\second}.
\end{align}
Zur Bestimmung der Zerfallskurve des nicht-angeregten Rhodiumisotops
$\ce{^{104}_{45}Rh}$ werden zunächst die Funktionswerte der für das angeregte
Isotop berechneten Funktion \eqref{eqn:RhodiumL} an der jeweiligen Stelle
$t_i \leq t^*$
abgezogen.
Es wird ein
\begin{align}
  t_\text{max} = \SI{320}{\second}
\end{align}
gewählt, ab welchem die Messpunkte abgeschnitten werden, da die Werte an den
darauffolgenden Stellen durch die statistischen Schwankungen nach Abzug
der Funktionsgleichung negativ sind.
Mit den logarithmierten übrigen Messdaten wird eine lineare Regression
durchgeführt, bei der folgende Werte ausgegeben werden:
\begin{align}
  m_\text{R,K} & = -\num{0.015(1)} \\
  b_\text{R,K} & = \num{6.7(3)}.
\end{align}
Es folgen die Parameter der Funktionsgleichung des nicht-angeregten
Rhodiumisotops
\begin{align}
  N_\text{0,R,K}(1-\symup{e}^{-\lambda_\text{R,K}\increment t}) =
  \symup{e}^{b_\text{R,K}} = \num{81(21)e1}
\end{align}
und
\begin{align}
  \lambda_\text{R,K} = -m_\text{R,K} = \num{0.015(1)}.
\end{align}
Daraus ergibt sich die Halbwertszeit des Rhodiumisotops $\ce{^{104}_{45}Rh}$
mit \eqref{eqn:RhodiumL}
\begin{align}
  T_{\frac{1}{2},\text{Rh,K}} = \SI{46(4)}{\second}.
\end{align}
In Abbildung \ref{fig:RhodiumE} sind die berechneten Zerfallskurven und die
summierte Zerfallskurve dargestellt.
An der Stelle $t^*$ lauten die
Funtkionswerte
\begin{align}
  N_{\increment t,\text{R,K}} (t^*) = \num{2(1)}
\end{align}
und
\begin{align}
  N_{\increment t,\text{R,L}} (t^*) = \num{26(19)}.
\end{align}

\begin{figure}
  \centering
  \includegraphics{build/RhodiumM.pdf}
  \caption{Messwerte der Rhodiummessung und Grenze $t^*$ zur Unterteilung in
  zwei Zerfallskurven. Es sind die Impulse halblogarithmisch gegen die Zeit
  aufgetragen.}
  \label{fig:RhodiumM}
\end{figure}

\begin{figure}
  \centering
  \includegraphics{build/Rhodium2.pdf}
  \caption{Messwerte und Ausgleichsfunktion des angeregten Rhodiumisotops.
  Es sind die Impulse halblogarithmisch gegen die Zeit aufgetragen.}
  \label{fig:Rhodium2}
\end{figure}

\begin{figure}
  \centering
  \includegraphics{build/Rhodium.pdf}
  \caption{Messwerte und Ausgleichsfunktion des nicht-angeregten Rhodiumisotops.
  Es sind die Impulse halblogarithmisch gegen die Zeit aufgetragen.}
  \label{fig:Rhodium}
\end{figure}

\begin{figure}
  \centering
  \includegraphics{build/RhodiumE.pdf}
  \caption{Messwerte und Ausgleichsfunktionen des Rhodiumisotops. Es sind die
  Impulse halblogarithmisch gegen die Zeit aufgetragen.}
  \label{fig:RhodiumE}
\end{figure}

\begin{table}[h]
  \centering
  \begin{tabular}{S S}
    \toprule
    {$t/\si{\second}$} & {$P$}\\
    \midrule
    200  & 648\\
    400  & 487\\
    600  & 444\\
    800  & 381\\
    1000 & 366\\
    1200 & 316\\
    1400 & 287\\
    1600 & 275\\
    1800 & 252\\
    2000 & 262\\
    2200 & 245\\
    2400 & 220\\
    2600 & 209\\
    \bottomrule
  \end{tabular}
  \caption{Messwerte zur Untersuchung der Halbwertszeit und Zerfallskurve von
  $\ce{^{128}_{53}I}$.}
  \label{tab:Iod}
\end{table}

\begin{table}[h]
  \centering
  \begin{tabular}{S S}
    \toprule
    {$t/\si{\second}$} & {$P$}\\
    \midrule
    20  & 817 \\
    40  & 595 \\
    60  & 417 \\
    80  & 303 \\
    100  & 265 \\
    120  & 191 \\
    140  & 136 \\
    160  & 116 \\
    180  & 99 \\
    200 &  86 \\
    220 &  78 \\
    240 &  56 \\
    260 &  63 \\
    280 &  49 \\
    300 &  49 \\
    320 &  44 \\
    340 &  36 \\
    360 &  36 \\
    380 &  46 \\
    \bottomrule
  \end{tabular}
  \caption{Messwerte zur Untersuchung der Halbwertszeit und Zerfallskurve von
  $\ce{^{104}_{45}Rh}$.}
  \label{tab:Rhodium}
\end{table}

\begin{table}[h]
  \centering
  \begin{tabular}{S S}
    \toprule
    {$t/\si{\second}$} & {$P$}\\
    \midrule
    400 & 30 \\
    420 & 32 \\
    440 & 28 \\
    460 & 21 \\
    480 & 25 \\
    500 & 33 \\
    520 & 22 \\
    540 & 23 \\
    560 & 22 \\
    580 & 27 \\
    600 & 25 \\
    620 & 28 \\
    640 & 12 \\
    660 & 19 \\
    680 & 10 \\
    700 & 19 \\
    \bottomrule
  \end{tabular}
  \caption{Messwerte zur Untersuchung der Halbwertszeit und Zerfallskurve von
  $\ce{^{104i}_{45}Rh}$.}
  \label{tab:Rhodium2}
\end{table}
