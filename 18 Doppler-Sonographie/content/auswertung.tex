\section{Auswertung}
\label{sec:Auswertung}

\subsection{Strömungsgeschwindigkeit}

Mit Formel \eqref{eqn:???} wird der zum Prismawinkel $\theta$ passende Dopplerwinkel $\alpha$
bestimmt. Die Ergebnisse sind in Tabelle \ref{tab:doppler} zu sehen.
%alpha = np.array([80.06, 70.53, 54.74])


\begin{table}[h]
  \centering
  \begin{tabular}{S S}
    \toprule
    {$\theta\:/\:\si{\degree}$} & {$\alpha\:/\: \si{\degree}$}\\
    \midrule
    15 & 80.06\\
    30 & 70.53\\
    60 & 54.74\\
    \bottomrule
  \end{tabular}
  \caption{Prismawinkel $\theta$ und der dazu passende Dopplerwinkel $\alpha$.}
  \label{tab:doppler}
\end{table}

Mit den Dopplerwinkeln und Formel \eqref{eqn:???} wird dann aus den gemessenen Frequenzverschiebungen
$\Delta\nu$ an jedem Rohr die Geschwindigkeit bestimmt. Die bestimmten Strömungsgeschwindigkeiten
sind dann in den Tabellen \ref{tab:v16mm}, \ref{tab:v10mm} und \ref{tab:v7mm} für das jeweilige Rohr
mit unterschiedlichem Innendurchmesser $d$ zu sehen.

%Tabelle für copy and paste:
\begin{table}[h]
  \centering
  \begin{tabular}{S S S S}
    \toprule
    {$v\:/\:\si{\percent}$} & {$v_{\SI{15}{\degree}}\:/\: \si{\meter\per\second}$} & {$v_{\SI{30}{\degree}}\:/\: \si{\meter\per\second}$} &
    {$v_{\SI{60}{\degree}}\:/\: \si{\meter\per\second}$}\\
    \midrule
    40 & 0.271 & 0.189 & 0.160\\
    45 & 0.334 & 0.231 & 0.190\\
    50 & 0.391 & 0.277 & 0.238\\
    55 & 0.446 & 0.332 & 0.291\\
    60 & 0.514 & 0.398 & 0.335\\
    \bottomrule
  \end{tabular}
  \caption{Strömungsgeschwindigkeiten $v$ im Rohr mit $d=\SI{16}{\milli\meter}$.}
  \label{tab:v16mm}
\end{table}

\begin{table}[h]
  \centering
  \begin{tabular}{S S S S}
    \toprule
    {$v\:/\:\si{\percent}$} & {$v_{\SI{15}{\degree}}\:/\: \si{\meter\per\second}$} & {$v_{\SI{30}{\degree}}\:/\: \si{\meter\per\second}$} &
    {$v_{\SI{60}{\degree}}\:/\: \si{\meter\per\second}$}\\
    \midrule
    40 & 0.508 & 0.462 & 0.457\\
    45 & 0.589 & 0.567 & 0.528\\
    50 & 0.701 & 0.675 & 0.647\\
    55 & 0.795 & 0.759 & 0.756\\
    60 & 0.923 & 0.857 & 0.857\\
    \bottomrule
  \end{tabular}
  \caption{Strömungsgeschwindigkeiten $v$ im Rohr mit $d=\SI{10}{\milli\meter}$.}
  \label{tab:v10mm}
\end{table}

\begin{table}[h]
  \centering
  \begin{tabular}{S S S S}
    \toprule
    {$v\:/\:\si{\percent}$} & {$v_{\SI{15}{\degree}}\:/\: \si{\meter\per\second}$} & {$v_{\SI{30}{\degree}}\:/\: \si{\meter\per\second}$} &
    {$v_{\SI{60}{\degree}}\:/\: \si{\meter\per\second}$}\\
    \midrule
    40 & 1.051 & 1.038 & 0.893\\
    45 & 1.225 & 1.237 & 1.094\\
    50 & 1.447 & 1.396 & 1.282\\
    55 & 1.624 & 1.654 & 1.477\\
    60 & 1.851 & 1.879 & 1.713\\
    \bottomrule
  \end{tabular}
  \caption{Strömungsgeschwindigkeiten $v$ im Rohr mit $d=\SI{7}{\milli\meter}$.}
  \label{tab:v7mm}
\end{table}

In den folgenden Diagrammen ist nun $\Delta\nu / \cos(\alpha)$ als Funktion der Strömungsgeschwindigkeit $v$
aufgetragen. Dies wurde in den Abbildungen \ref{fig:plotvwinkel15}, \ref{fig:plotvwinkel30} und
\ref{fig:plotvwinkel60} aufgetragen und durch eine Ausgleichsgerade der lineare
Zusammenhang verdeutlicht.

\begin{figure}
  \centering
  \includegraphics[width = \textwidth]{build/plotvwinkel15.pdf}
  \caption{Zusammenhang von $v$ und $\Delta\nu / \cos(\alpha)$ bei Dopplerwinkel $\alpha = \SI{15}{\degree}$.}
  \label{fig:plotvwinkel15}
\end{figure}

\begin{figure}
  \centering
  \includegraphics[width = \textwidth]{build/plotvwinkel30.pdf}
  \caption{Zusammenhang von $v$ und $\Delta\nu / \cos(\alpha)$ bei Dopplerwinkel $\alpha = \SI{30}{\degree}$.}
  \label{fig:plotvwinkel30}
\end{figure}

\begin{figure}
  \centering
  \includegraphics[width = \textwidth]{build/plotvwinkel60.pdf}
  \caption{Zusammenhang von $v$ und $\Delta\nu / \cos(\alpha)$ bei Dopplerwinkel $\alpha = \SI{60}{\degree}$.}
  \label{fig:plotvwinkel60}
\end{figure}

\FloatBarrier

\subsection{Strömungsprofil}

Das Strömungsprofil wird am Rohr mit Durchmesser $d = \SI{10}{\milli\meter}$ aufgenommen.
Aus den Werten von $\Delta\nu$ wird wie im vorangegangenen Kapitel die Strömungsgeschwindigkeit
bestimmt.
Die Intensität wird direkt im Diagramm gegen die Tiefe aufgetragen.
Für die Geschwindigkeit $v = \SI{70}{\percent}$ sind die Diagramme in den
Abbildungen \ref{fig:vprofil70} und \ref{fig:intprofil70} zu sehen.
Die Abbildungen \ref{fig:vprofil45} und \ref{fig:intprofil45} passen dann zur
Geschwindigkeit $v = \SI{45}{\percent}$.

\begin{figure}
  \centering
  \includegraphics[width = \textwidth]{build/plotvprofil70.pdf}
  \caption{Geschwindigkeitsprofil bei $v = \SI{70}{\percent}$.}
  \label{fig:vprofil70}
\end{figure}

\begin{figure}
  \centering
  \includegraphics[width = \textwidth]{build/plotintensprofil70.pdf}
  \caption{Intensitätsprofil bei $v = \SI{70}{\percent}$.}
  \label{fig:intprofil70}
\end{figure}

\begin{figure}
  \centering
  \includegraphics[width = \textwidth]{build/plotvprofil45.pdf}
  \caption{Geschwindigkeitsprofil bei $v = \SI{45}{\percent}$.}
  \label{fig:vprofil45}
\end{figure}

\begin{figure}
  \centering
  \includegraphics[width = \textwidth]{build/plotintensprofil45.pdf}
  \caption{Intensitätsprofil bei $v = \SI{45}{\percent}$.}
  \label{fig:intprofil45}
\end{figure}
