\section{Diskussion}
\label{sec:Diskussion}

\subsection{Strömungsgeschwindigkeit}

Die Strömungsgeschwindigkeit konnte unter verschiedenem Winkel bei jedem Rohr
zufriedenstellend bestimmt werden. Beachtenswert ist, dass die Abweichung der
Messreihen unter verschiedenem Winkel bei höherer allgemeiner Geschwindigkeit,
also geringerem Rohrdurchmesser kleiner werden. Dieses Verhalten lässt sich so erklären,
dass dort kleinere Schwankungen in der Anzeige einen geringeren Beitrag leisten.

Mit den Diagrammen von $\Delta\nu / \cos(\alpha)$ in Abhängigkeit von
$v$ konnte ein sehr gut erkennbarer linearer Zusammenhang gezeigt werden.

\subsection{Strömungsprofil}

Nach der Anleitung ist die Vorlaufstrecke im Prisma $\SI{30.7}{\milli\meter}$
lang. Mit den Angaben für die Messtiefe in Acryl sollte das Signal also nach
$\SI{12.28}{\second}$ auf das Rohr treffen. Dies ist in allen Diagrammen bezüglich
des Strömungsprofils gut zu erkennen, da immer dort Flanken beginnen oder fallen.

Die eigentliche Breite des Rohres ist allerdings nicht gut zu erkennen. Theoretisch müsste
mit den Angaben der Anleitung die Breite des klar zu erkennenden Bergs $\SI{6.67}{\micro\second}$
betragen. Zu erkennen ist aber eher eine Breite von Tiefpunkt zu Tiefpunkt von
$\SI{5}{\micro\second}$, was einem Innendurchmesser von ca. $\SI{7.5}{\milli\meter}$
entspräche.
