\section{Theorie}
\label{sec:Theorie}

Die Art und Weise wie sich ein Körper durch eine Flüssigkeit
bewegt hängt von einigen Faktoren ab. Diese Faktoren sind unter anderem
 die Beschaffenheit des Körpers wie Masse und Volumen, die stark temperaturabhängige
Viskosität und die Dichte der Flüssigkeit.
Die  Viskosität kann mit Hilfe eines Kugelfallviskosimeters bestimmt werden.
Dort wird eine Kugel durch ein Rohr mit leicht größerem Radius fallen gelassen.
Es muss darauf geachtet werden dieses Experiment bei einer laminaren
Strömung durchzuführen. Das heißt, dass sich die Flüssigkeitsschichten nicht
vermischen, sich also keine Turbulenzen bilden.
Eine Kenngröße die angibt, ob mit laminarer oder turbulenter Strömung
zu rechnen ist, ist die $Reynolds-Zahl~Re$. Berechnet wird sie mit Formel
\eqref{eqn:reynolds}; wobei $d$ der Durchmesser des Rohres ist und $v$
die Geschwindigkeit der Kugel.

\begin{equation}
  Re = \frac{\rho v d}{\eta}
  \label{eqn:reynolds}
\end{equation}

Der kritische Wert, über dem bei Rohrströmungen mit Turbulenzen zu rechnen
ist, ist $Re_{\text{krit}} = 2300$.
Die Kräfte, die auf die Kugel wirken sind dann: die Gewichtskraft $\vec{F_{\text{g}}}$
, der Auftrieb $\vec{F_{\text{A}}}$ und die Reibung $\vec{F_{\text{R}}}$.
Die Reibung ist proportional zur Geschwindigkeit der Kugel und nimmt zu, bis
sich ein Gleichgewicht einstellt und die Kugel sich mit konstanter Geschwindigkeit
weiterbewegt.
Wenn man dann die Fallzeit t bestimmt, kann die Viskosität $\eta$ mit Formel
\eqref{eqn:visko1} bestimmt werden.

\begin{equation}
  \eta = K (\rho_{\text{K}}-\rho_{\text{Fl}}) \cdot t
  \label{eqn:visko1}
\end{equation}

Bei vielen Flüssigkeiten ist $\eta$ abhängig von der Temperatur $T$.

Dann
wird die $Andradesche~Gleichung$ \eqref{eqn:andra} genutzt.

\begin{equation}
  \eta(T) = \symup{A} \symup{e}^{\frac{\symup{B}}{T}}
  \label{eqn:andra}
\end{equation}

$\symup{A}$ und $\symup{B}$ sind Konstanten.
