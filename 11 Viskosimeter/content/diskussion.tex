\section{Diskussion}
\label{sec:Diskussion}

\subsection{Temperaturabhängigkeit der Viskosität}

Trotz der vielen Messungenauigkeiten, beispielsweise bei der Fallzeit $t$, oder
dem Gewicht $m$ und dem Radius $r$ der Kugeln, sind die Messfehler der
Koeffizienten in der Andraedschen Gleichung klein.
Die prozentualen Abweichungen $\Delta x$ der Viskositäts-Werte
$\eta_\text{mess}(T)$ von den
Literaturwerten $\eta_\text{lit}(T)$ \cite{Viskositätliteratur}
bei verschiedenen Temperaturen ist ebenfalls gering,
wie man in Tabelle \ref{tab:VglTemperatur} erkennen kann.
Kleine Fehler lassen sich dadurch erklären, dass sich während des
Temperaturanstiegs Gasblasen im Fallrohr gebildet haben, welche die Kugel
aufgehalten haben.

\begin{table}[h]
  \centering
  \caption{Vergleich der Messwerte mit Literaturwerten.}
  \label{tab:VglTemperatur}
  \begin{tabular}{c c c c}
    \toprule
    $T/\si{\celsius}$ & $\eta_\text{mess}/\si{\gram\per\meter\per\second}$ &
    $\eta_\text{lit}/\si{\gram\per\meter\per\second}$ &
    $\Delta x/\si{\percent}$ \\
    \midrule
    20 & 1.174 & 1.002 & 14.7 \\
    30 & 0.980 & 0.798 & 18.6 \\
    40 & 0.827 & 0.653 & 21.0 \\
    50 & 0.705 & 0.547 & 22.4 \\
    60 & 0.607 & 0.467 & 23.2 \\
    70 & 0.527 & 0.404 & 23.4 \\
    \bottomrule
  \end{tabular}
\end{table}

\subsection{Reynolds-Zahl}
