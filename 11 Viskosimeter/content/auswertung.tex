\section{Auswertung}
\label{sec:Auswertung}

\subsubsection{Viskosität bei Raumtemperatur}

Bei der kleinen Glaskugel werden folgende Werte für Gewicht und Radius
gemessen:
\begin{align}
  m_\text{k} & = \SI{4.4531e-03}{\kilo\gram} & r_\text{k} & =
  \SI{7.8203(9)e-03}{\meter}.
\end{align}
Daraus folgt die Dichte

\begin{equation}
  \rho_\text{k} = \frac{m_\text{k}}{V_\text{k}} = \frac{m_\text{k}}
  {\frac{4}{3}\pi r_\text{k}^3} = \SI{2222.9(7)}{\kilo\gram\per\cubic\meter}.
\end{equation}
In Tabelle \ref{tab:klKugel} sind die Messwerte der Fallzeit der kleinen
Kugel aufgeführt. Diese werden gemittelt, sodass mit einer Zeit

\begin{equation}
  t_{k} = \SI{12.9(1)}{\second}
\end{equation}
weitergerechnet werden kann.
Mit der gegebenen Apparaturkonstante

\begin{equation}
  K_\text{k} = \SI{0.07640e-6}{\pascal\cubic\meter\per\kilo\gram}
\end{equation}
für die kleine Kugel und der Dichte von Wasser \cite{Wasserdichte}

\begin{equation}
  \rho_\text{W} = \SI{998.2}{\kilo\gram\per\cubic\meter}
\end{equation}
folgt bei Raumtemperatur $T=\SI{291.15}{\kelvin}$ über die Formel

\begin{equation}
  \eta_\text{W} = (\rho_\text{k}-\rho_\text{W})K_\text{k} \cdot t_\text{k}
  \label{eqn:Viskositaet}
\end{equation}
die Viskosität des destillierten Wassers

\begin{equation}
  \eta_\text{W} = \SI{1.21(1)e-3}{\pascal\second}.
  \label{eqn:Viskosi}
\end{equation}

\subsubsection{Apparaturkonstante}

Die Messung der Temperaturabhängigkeit wird mit der großen Kugel durchgeführt,
weshalb zunächst die Apparaturkonstante für diese Kugel bestimmt werden muss.
Die Werte für die Masse und den Radius der großen Kugel betragen

\begin{align}
  m_\text{g} & = \SI{4.96e-3}{\kilo\gram} & r_\text{g} & =
  \SI{7.908(1)e-3}{\meter}.
\end{align}
Das führt zu der Dichte

\begin{equation}
  \rho_\text{g} = \frac{m_\text{g}}{V_\text{g}} = \frac{m_\text{g}}
  {\frac{4}{3}\pi r_\text{g}^3} = \SI{2394(1)}{\kilo\gram\per\cubic\meter}.
  \label{eqn:rhog}
\end{equation}
In Tabelle \ref{tab:grKugel} sind die Messwerte der Fallzeit der großen
Kugel aufgeführt. Der Mittelwert der zehn Werte beträgt

\begin{equation}
  t_\text{g} = \SI{84.1(4)}{\second}.
\end{equation}
Über die berechneten Dichten, die Fallzeit und die Viskosität $\eta_\text{W}$
von Wasser folgt bei Raumtemperatur die Apparaturkonstante

\begin{equation}
  K_\text{g} = \frac{\eta_\text{W}}{t(\rho_\text{g}-\rho_\text{W})}
  = \SI{1.03(1)e-8}{\pascal\cubic\meter\per\kilo\gram}
\end{equation}
für die große Kugel.

\subsubsection{Temperaturabhängigkeit der Viskosität}

In Tabelle \ref{tab:FallTemp} sind unter anderem die Messwerte für die Fallzeit
bei verschiedenen Temperaturen abgebildet.
Aus den Werten für die Fallzeit $t$ kann über die Formel \eqref{eqn:Viskositaet}
die Viskosität $\eta(T)$ für die jeweiligen Temperaturen $T$ bestimmt werden.
Die Messwertpaare werden dann durch eine Exponentialfunktion, wie in der
Andraedschen Gleichung \eqref{eqn:andra}, angenähert.
In Abbildung \ref{fig:VisTemp} sind die Messwerte und die Ausgleichsfunktion
jeweils mit ln$(\eta)$ und $\frac{1}{T}$ dargestellt.
Die Werte für $\eta$, ln$(\eta)$ und $\frac{1}{T}$ sind in Tabelle
\ref{tab:FallTemp} abgebildet.
Mit einer Ausgleichsrechnung ergeben sich die Koeffizienten aus Gleichung
\eqref{eqn:andra}

\begin{equation}
  A = \num{4.8(5)e-6}
\end{equation}
und

\begin{equation}
  B = (\num{1610(30)}).
\end{equation}
Damit lautet die Funktion der Viskosität von der Temperatur

\begin{equation}
  \eta(T) = (\num{4.8(5)e-6}) \symup{e}^{\frac{\num{1610(30)}}{T}\si{\second}}
  \si{\pascal\second}.
\end{equation}
Ein paar Werte dieser Funktion für verschiedene Temperaturen sind in
Tabelle \ref{tab:Viskovgl} aufgeführt.

\subsubsection{Reynolds-Zahl}

Die Reynolds-Zahl des destillierten Wassers kann mit Hilfe der Viskosität aus
Gleichung \eqref{eqn:Viskosi}, der Dichte der großen Kugel aus Gleichung
\eqref{eqn:rhog},
der Fallrohrdicke, die etwa dem Durchmesser der großen Kugel

\begin{equation}
  d_\text{g} = 2 r_\text{g} = \SI{15.816(2)e-3}{\meter}
\end{equation}
entspricht, und der Geschwindigkeit der großen Kugel

\begin{equation}
  v_\text{g} = \frac{0.1}{38.09}\si{\meter\per\second} =
  \SI{2.625e-3}{\meter\per\second}
\end{equation}
bestimmt werden.
Mit Gleichung \eqref{eqn:reynolds} folgt

\begin{equation}
  Re = \num{82.2(8)}.
\end{equation}

\begin{figure}
  \centering
  \includegraphics[width=\textwidth]{plotandrat.pdf}
  \caption{Graph von $\eta(T)$ und Messwerte.}
  \label{fig:VisTemp}
\end{figure}

\begin{table}[h]
  \centering
  \caption{Messwerte der Fallzeit der kleinen Kugel.}
  \label{tab:klKugel}
  \begin{tabular}{c c}
    \toprule
    $t/\si{\second}$ \\
    \midrule
    12.84 \\
    12.96 \\
    12.81 \\
    12.96 \\
    13.03 \\
    13.07 \\
    12.76 \\
    12.78 \\
    12.93 \\
    13.03 \\
    \bottomrule
  \end{tabular}
\end{table}

\begin{table}[h]
  \centering
  \caption{Messwerte der Fallzeit der großen Kugel.}
  \label{tab:grKugel}
  \begin{tabular}{c c}
    \toprule
    $t/\si{\second}$ \\
    \midrule
    84.01 \\
    83.62 \\
    84.13 \\
    84.63 \\
    83.87 \\
    83.94 \\
    84.44 \\
    84.36 \\
    84.03 \\
    83.50 \\
    \bottomrule
  \end{tabular}
\end{table}

\begin{table}[h]
  \centering
  \caption{Messwerte für die Temperaturmessung mit der großen Kugel.}
  \label{tab:Viskovgl}
  \begin{tabular}{c c}
    \toprule
    $T/\si{\celsius}$ & $t/\si{\second}$ \\
    \midrule
    18.0 & 84.13 \\
    18.0 & 84.63 \\
    40.0 & 58.50 \\
    40.0 & 58.55 \\
    45.0 & 54.56 \\
    45.0 & 55.58 \\
    48.0 & 50.84 \\
    48.0 & 50.97 \\
    51.0 & 47.35 \\
    51.0 & 47.63 \\
    54.0 & 44.75 \\
    54.0 & 44.38 \\
    58.0 & 42.58 \\
    58.0 & 42.40 \\
    61.0 & 40.79 \\
    61.0 & 40.76 \\
    65.5 & 38.10 \\
    65.5 & 38.09 \\
    70.5 & 38.50 \\
    70.5 & 38.67 \\
    \bottomrule
  \end{tabular}
\end{table}

\begin{table}[h]
  \centering
  \caption{Messwerte für die Temperaturmessung mit der großen Kugel.}
  \label{tab:FallTemp}
  \begin{tabular}{c c}
    \toprule
    $T/\si{\celsius}$ & $\eta(T)/\si{\gram\per\meter\per\second}$ \\
    \midrule
    20 & 1.174 \\
    30 & 0.980 \\
    40 & 0.827 \\
    50 & 0.705 \\
    60 & 0.607 \\
    70 & 0.527 \\
    \bottomrule
  \end{tabular}
\end{table}
