\section{Auswertung}
\label{sec:Auswertung}

\subsubsection{Viskosität bei Raumtemperatur}

Bei der kleinen Glaskugel werden folgende Werte für Gewicht und Radius
gemessen:
\begin{align}
  m_\text{k} & = \SI{4.4531}{\gram} & r_\text{k} & =
  \SI{7.8203(9)}{\milli\meter}.
\end{align}
Daraus folgt die Dichte

\begin{equation}
  \rho_\text{k} = \frac{m_\text{k}}{V_\text{k}} = \frac{m_\text{k}}
  {\frac{4}{3}\pi r_\text{k}^3} = \SI{2222.9(7)}{\kilo\gram\per\cubic\meter}.
\end{equation}
Die zehn Werte der Fallzeit der kleinen Kugel am Viskosimeter werden gemitteln,
sodass mit einer Zeit

\begin{equation}
  t_{k} = \SI{12.9(1)}{\second}
\end{equation}
weitergerechnet werden kann.
Mit der gegebenen Apparaturkonstante

\begin{equation}
  K_\text{k} = \SI{0.07640}{\pascal\cubic\centi\meter\per\kilo\gram}
\end{equation}
für die kleine Kugel und der Dichte von Wasser \cite{Wasserdichte}

\begin{equation}
  \rho_\text{W} = \SI{998.2}{\kilo\gram\per\cubic\meter}
\end{equation}
folgt bei Raumtemperatur $T=\SI{291.15}{\kelvin}$ über die Formel

\begin{equation}
  \eta_\text{W} = (\rho_\text{k}-\rho_\text{W})K_\text{k} \cdot t_\text{k}
  \label{eqn:Viskositaet}
\end{equation}
die Viskosität des destillierten Wassers

\begin{equation}
  \eta_\text{W} = \SI{1.21(1)}{\milli\pascal\second}.
  \label{eqn:Viskosi}
\end{equation}

\subsubsection{Apparaturkonstante}

Die Messung der Temperaturabhängigkeit wird mit der großen Kugel durchgeführt,
weshalb zunächst die Apparaturkonstante für diese Kugel bestimmt werden muss.
Die Werte für die Masse und den Radius der großen Kugel betragen

\begin{align}
  m_\text{g} & = \SI{4.96}{\gram} & r_\text{g} & = \SI{7.908(1)}{\milli\meter}.
\end{align}
Das führt zu der Dichte

\begin{equation}
  \rho_\text{g} = \frac{m_\text{g}}{V_\text{g}} = \frac{m_\text{g}}
  {\frac{4}{3}\pi r_\text{g}^3} = \SI{2394(1)}{\kilo\gram\per\cubic\meter}.
  \label{eqn:rhog}
\end{equation}
Der Mittelwert der zehn Fallzeiten der großen Kugel beträgt

\begin{equation}
  t_\text{g} = \SI{84.1(4)}{\second}.
\end{equation}
Über die berechneten Dichten, die Fallzeit und die Viskosität $\eta_\text{W}$
von Wasser folgt bei Raumtemperatur die Apparaturkonstante

\begin{equation}
  K_\text{g} = \frac{\eta_\text{W}}{t(\rho_\text{g}-\rho_\text{W})}
  = \SI{10.3(1)}{\milli\pascal\cubic\centi\meter\per\kilo\gram}
\end{equation}
für die große Kugel.

\subsubsection{Temperaturabhängigkeit der Viskosität}

In Tabelle \ref{tab:FallTemp} sind die Messwerte für die Fallzeit
bei verschiedenen Temperaturen abgebildet.

\begin{table}[h]
  \centering
  \caption{Messwerte der Temperatur und Fallzeit.}
  \label{tab:FallTemp}
  \begin{tabular}{c c}
    \toprule
    $T/\si{\celsius}$ & $t/\si{\second}$ \\
    \midrule
    18.0 & 84.13 \\
    18.0 & 84.63 \\
    40.0 & 58.50 \\
    40.0 & 58.55 \\
    45.0 & 54.56 \\
    45.0 & 55.58 \\
    48.0 & 50.84 \\
    48.0 & 50.97 \\
    51.0 & 47.35 \\
    51.0 & 47.63 \\
    54.0 & 44.75 \\
    54.0 & 44.38 \\
    58.0 & 42.58 \\
    58.0 & 42.40 \\
    61.0 & 40.79 \\
    61.0 & 40.76 \\
    65.5 & 38.10 \\
    65.5 & 38.09 \\
    70.5 & 38.50 \\
    70.5 & 38.67 \\
    \bottomrule
  \end{tabular}
\end{table}

Aus den Werten für die Fallzeit $t$ kann über die Formel \eqref{eqn:Viskositaet}
die Viskosität $\eta(T)$ für die jeweiligen Temperaturen $T$ bestimmt werden.
Die Messwertpaare werden dann durch eine Exponentialfunktion, wie in der
Andraedschen Gleichung \eqref{eqn:andra}, angenähert.
In Abbildung \ref{fig:VisTemp} sind die Messwerte und die Ausgleichsfunktion
jeweils mit $ln(\eta)$ und $\frac{1}{T}$ dargestellt.
Daraus ergeben sich die Koeffizienten

\begin{equation}
  A = \num{4.8(5)e-6}
\end{equation}
und

\begin{equation}
  B = (\num{1610(30)}).
\end{equation}
Damit lautet die Funktion der Viskosität von der Temperatur

\begin{equation}
  \eta(T) = (\num{4.8(5)}) \symup{e}^{\frac{\num{1610(30)}}{T}\si{\second}}
  \si{\micro\pascal\second}
\end{equation}

\begin{figure}
  \centering
  \includegraphics[width=\textwidth]{plotandrat.pdf}
  \caption{Graph von $\eta(T)$ und Messwerte.}
  \label{fig:VisTemp}
\end{figure}

\subsubsection{Reynolds-Zahl}

Die Reynolds-Zahl des destillierten Wassers kann mit Hilfe der Viskosität aus
Gleichung \eqref{eqn:Viskosi}, der Dichte der großen Kugel aus Gleichung
\eqref{eqn:rhog},
der Fallrohrdicke, die etwa dem Durchmesser der großen Kugel

\begin{equation}
  d_\text{g} = 2 r_\text{g} = \SI{15.816(2)}{\milli\meter}
\end{equation}
entspricht, und der Geschwindigkeit der großen Kugel

\begin{equation}
  v_\text{g} = \frac{0.1}{\num{84.1(0.4)}}\si{\meter\per\second} =
  \SI{1.190(5)e-3}{\meter\per\second}
\end{equation}
bestimmt werden.
Mit Gleichung \eqref{eqn:reynolds} folgt

\begin{equation}
  Re = \num{37.3(4)}.
\end{equation}
